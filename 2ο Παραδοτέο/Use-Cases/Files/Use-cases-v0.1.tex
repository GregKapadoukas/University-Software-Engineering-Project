%! TeX program = lualatex
\documentclass[12pt,a4paper]{article}

\usepackage[nil]{babel}
\usepackage{unicode-math}
\usepackage[svgnames]{xcolor}
\usepackage{lmodern}
\usepackage{graphicx}
\usepackage{wrapfig}
\usepackage{float}
\usepackage{parskip}
\usepackage{enumitem}

\makeatother
\babelprovide[import=el, main, onchar=ids fonts]{greek} % can also do import=el-polyton
\babelprovide[import, onchar=ids fonts]{english}

\babelfont{rm}
          [Language=Default]{Liberation Sans}
\babelfont[english]{rm}
          [Language=Default]{Liberation Sans}
\babelfont{sf}
          [Language=Default]{Liberation Sans}
\babelfont{tt}
          [Language=Default]{Liberation Sans}

%Enter Title Here
\title{Use-cases-v0.1 \\ LibShare}
\author{\textbf{Ονόματα / ΑΜ / Έτος:} \\ Γρηγόρης Καπαδούκας / 1072484 / 4\textdegree \\ Χρήστος Μπεστητζάνος / 1072615 / 4\textdegree \\ Νικόλαος Αυγέρης / 1067508 / 5\textdegree \\ Περικλής Κοροντζής / 1072563 / 4\textdegree}

\begin{document}

\makeatletter
\begin{center}
	\LARGE{\@title} \\
	\pagebreak
	\begin{LARGE}\@author\end{LARGE} \\
\end{center}
\pagebreak

%Insert Body Here
\section{Use Case Diagram}

\section{Ροές των Use Cases}

\subsection{Ενοικίαση βιβλίου από άλλο χρήστη}
\subsubsection*{Βασική Ροή <<Ενοικίαση Βιβλίου>>: Συναλλαγή από κοντά:}
\begin{enumerate}
    \item Το σύστημα ελέγχει αν υπάρχουν διαθέσιμα αναρτημένα βιβλία από άλλους χρήστες που προτίθενται για ενοικίαση, και διαπιστώνει ότι υπάρχουν.
    \label{Έλεγχος ύπαρξης βιβλίου}
    \item To σύστημα εμφανίζει τα διαθέσιμα βιβλία στον χρήστη.
    \item Ο χρήστης επιλέγει ένα από τα βιβλία.
    \item Το σύστημα εμφανίζει στον χρήστη πληροφορίες για τους διαθέσιμους ιδιοκτήτες που προσφέρουν το βιβλίο, διαθέσιμους τρόπους συναλλαγής (ταχυδρομικώς ή από κοντά) και την τιμή ανά μέρα.
    \item Ο ενοικιαστής κάνει αίτηση ενοικίασης στον ιδιοκτήτη που επέλεξε για το βιβλίο που θέλει, επιλέγοντας συναλλαγή από κοντά.
    \label{Επιλογή τρόπου συναλλαγής}
    \item Το σύστημα επιβεβαιώνει ότι ο ενοικιαστής έχει αρκετά χρήματα στο λογαριασμό του για να καλύψει το "ποσό ασφαλείας" που θα δεσμευτεί αργότερα από τον λογαριασμό του.
    \label{Έλεγχος ποσού ασφαλείας}
    \item Το σύστημα επιβεβαιώνει ότι η αίτηση ενοικίασης έχει γίνει αποδεκτή από τον ιδιοκτήτη.
    \label{Αποδοχή ή απόρριψη συναλλαγής}
    \item Το σύστημα εμφανίζει στοιχεία επικοινωνίας του ιδιοκτήτη στον ενοικιαστή.
    \item Ο ενοικιαστής, αφού βρεθεί με τον ιδιοκτήτη από κοντά και παραλάβει το βιβλίο, ενημερώνει το σύστημα ότι έχει κατοχή του βιβλίου. Ο ιδιοκτήτης επίσης ενημερώνει το σύστημα.
    \item Το σύστημα ελέγχει αν έλαβε ενημέρωση κατοχής του βιβλίου και από τον ενοικιαστή και από τον ιδιοκτήτη.
    \label {Δεν ενημερώνεται η κατοχή}
    \item Το σύστημα δεσμεύει "ποσό ασφαλείας" από τον λογαριασμό του ενοικιαστή, και αρχίζει να τον χρεώνει αυτόματα μέρα με τη μέρα.
    \label{Τέλος dispute resolved - Τέλος χρημάτων}
    \item Ο ενοικιαστής, αφού τελειώσει το βιβλίο, βρίσκεται πάλι με τον ιδιοκτήτη και του παραχωρεί το βιβλίο, έπειτα του οποίου ενημερώνουν και οι δύο το σύστημα ότι το βιβλίο έχει επιστραφεί.
    \label{Επιστροφή βιβλίου - Τέλος λεφτά δεν φτάνουν}
    \item Το σύστημα, λάβει ενημέρωση επιστροφής βιβλίου και από τα δύο μέλη, σταματάει να χρεώνει τον ενοικιαστή και του επιστρέφει το "ποσό ασφαλείας".
    \label{Τέλος ενοικίασης}
\end{enumerate}

\subsubsection*{Εναλλακτική Ροή 1: Δεν υπάρχουν Βιβλία προς ενοικίαση:}
\begin{enumerate}
    \item[\ref{Έλεγχος ύπαρξης βιβλίου}.1.] Το σύστημα διαπιστώνει ότι δεν υπάρχουν διαθέσιμα βιβλία για τον ενοικιαστή να νοικιάσει.
    \item[\ref{Έλεγχος ύπαρξης βιβλίου}.2.] Το σύστημα ενημερώνει τον χρήστη ότι δεν υπάρχουν διαθέσιμα βιβλία.
\end{enumerate}

\subsubsection*{Εναλλακτική Ροή 2: Η συναλλαγή γίνεται ταχυδρομικώς:}
\begin{enumerate}
    \item[\ref{Επιλογή τρόπου συναλλαγής}.1.] Ο ενοικιαστής κάνει αίτηση ενοικίασης στον ιδιοκτήτη που επέλεξε για το βιβλίο που θέλει, επιλέγοντας συναλλαγή ταχυδρομικώς.
    \item[\ref{Επιλογή τρόπου συναλλαγής}.2.]Το σύστημα επιβεβαιώνει ότι ο ενοικιαστής έχει αρκετά χρήματα στο λογαριασμό του για να καλύψει το "ποσό ασφαλείας" που θα δεσμευτεί αργότερα από τον λογαριασμό του.
    \item[\ref{Επιλογή τρόπου συναλλαγής}.3.] Το σύστημα επιβεβαιώνει ότι η αίτηση ενοικίασης έχει γίνει αποδεκτή από τον ιδιοκτήτη και ότι έχει γίνει η αποστολή.
    \item[\ref{Επιλογή τρόπου συναλλαγής}.4.] Το σύστημα εμφανίζει τα στοιχεία του ιδιοκτήτη και tracking number από την αποστολή στον ενοικιαστή και γίνεται δέσμευση του "ποσού ασφαλείας" από τον λογαριασμό του ενοικιαστή.
    \item[\ref{Επιλογή τρόπου συναλλαγής}.5.] Ο ενοικιαστής, αφού παραλάβει το βιβλίο ταχυδρομικώς, ενημερώνει το σύστημα ότι έχει κατοχή του βιβλίου (η ενημέρωση κατοχής από τον αποστολέα γίνεται αυτόματα μέσω της κατάστασης δέματος με χρήση του tracking number).
    \item[\ref{Επιλογή τρόπου συναλλαγής}.6.] Το σύστημα, αφού λάβει ενημέρωση κατοχής του βιβλίου και από τον ιδιοκτήτη, αρχίζει να τον χρεώνει μέρα με τη μέρα.
    \item[\ref{Επιλογή τρόπου συναλλαγής}.7.] Ο ενοικιαστής, αφού τελειώσει το βιβλίο, στέλνει πίσω στον ιδιοκτήτη το βιβλίο ταχυδρομικώς και ενημερώνει τη πλατφόρμα με tracking number (το οποίο θα χρησιμοποιηθεί για να ανανεωθεί η κατάσταση κατοχής αυτόματα).
    \item[\ref{Επιλογή τρόπου συναλλαγής}.9.] Η περίπτωση χρήσης συνεχίζεται από το βήμα \ref{Τέλος ενοικίασης} της βασικής ροής.
\end{enumerate}

\subsubsection*{Εναλλακτική Ροή 3: Η συναλλαγή απορρίπτεται από τον ιδιοκτήτη:}
\begin{enumerate}
    \item[\ref{Έλεγχος ποσού ασφαλείας}\|\ref{Αποδοχή ή απόρριψη συναλλαγής}.1, \ref{Επιλογή τρόπου συναλλαγής}.2\|3.1.] Το σύστημα παρατηρεί ότι η αίτηση ενοικίασης απορρίπτεται από τον ιδιοκτήτη, ή ότι δεν έχει ο ενοικιαστής αρκετό χρηματικό ποσό στο λογαριασμό για να καλύψει το "ποσό ασφαλείας".
    \item[\ref{Έλεγχος ποσού ασφαλείας}\|\ref{Αποδοχή ή απόρριψη συναλλαγής}.2, \ref{Επιλογή τρόπου συναλλαγής}.2\|3.2.] Ο ενοικιαστής ενημερώνεται από το σύστημα ότι η αίτησή του απορρίφθηκε.
\end{enumerate}

\subsubsection*{Εναλλακτική Ροή 4: Δεν ενημερώνεται και από τους δύο η κατάσταση κατοχής - Επίλυση προβλήματος:}
\begin{enumerate}
    \item[\ref{Δεν ενημερώνεται η κατοχή}\|\ref{Επιστροφή βιβλίου - Τέλος λεφτά δεν φτάνουν}.1, 5.6.1] Το σύστημα λαμβάνει ενημέρωση κατοχής από τον ιδιοκτήτη και όχι από τον ενοικιαστή, ή αντίστροφα.
    \item[\ref{Δεν ενημερώνεται η κατοχή}\|\ref{Επιστροφή βιβλίου - Τέλος λεφτά δεν φτάνουν}.2, 5.6.2] Το σύστημα ενημερώνει και τα δύο μέλη να επικοινωνήσουν μεταξύ τους και αν δεν λυθεί το θέμα να επικοινωνήσουν με την υποστήριξη πελατών.
    \item[\ref{Δεν ενημερώνεται η κατοχή}\|\ref{Επιστροφή βιβλίου - Τέλος λεφτά δεν φτάνουν}.3, 5.6.3] Το σύστημα λαμβάνει ενημέρωση κατοχής και από τον δεύτερο χρήστη.
    \item[\ref{Δεν ενημερώνεται η κατοχή}\|\ref{Επιστροφή βιβλίου - Τέλος λεφτά δεν φτάνουν}.4, 5.6.4] Η περίπτωση χρήσης συνεχίζεται από το βήμα \ref{Τέλος dispute resolved - Τέλος χρημάτων} της βασικής ροής αν βρισκόταν στην αρχική ενοικίαση, και στο βήμα \ref{Τέλος ενοικίασης} αν βρισκόταν στην φάση επιστροφής του βιβλίου.
\end{enumerate}

\subsubsection*{Εναλλακτική Ροή 5: Δεν ενημερώνεται και από τους δύο η κατάσταση κατοχής - Επίλυση από υποστήριξη πελατών:}
\begin{enumerate}
    \item[\ref{Δεν ενημερώνεται η κατοχή}\|\ref{Επιστροφή βιβλίου - Τέλος λεφτά δεν φτάνουν}.3.1, 5.6.3.1] Το σύστημα δεν λαμβάνει ενημέρωση κατοχής και από τον δεύτερο χρήστη.
    \item[\ref{Δεν ενημερώνεται η κατοχή}\|\ref{Επιστροφή βιβλίου - Τέλος λεφτά δεν φτάνουν}.3.2, 5.6.3.2]Το σύστημα ενημερώνεται από την υποστήριξη πελατών και το "ποσό ασφαλείας" αποστέλλεται στον λογαριασμό του χρήστη που αποφάσισε η υποστήριξη πελατών ότι αδικήθηκε.
\end{enumerate}

\subsubsection*{Εναλλακτική Ροή 6: Τέλος χρημάτων ενοικιαστή - Μετέπειτα προσθήκη απαραίτητου ποσού:}
\begin{enumerate}
    \item[\ref{Τέλος dispute resolved - Τέλος χρημάτων}.1.] Το σύστημα αντιλαμβάνεται ότι έχουν τελειώσει τα χρήματα του ενοικιαστή.
    \item[\ref{Τέλος dispute resolved - Τέλος χρημάτων}.2.] Το σύστημα ζητάει από τον ενοικιαστή να προσθέσει παραπάνω χρήματα στον λογαριασμό του.
    \item[\ref{Τέλος dispute resolved - Τέλος χρημάτων}.3.] Ο ενοικιαστής προσθέτει παραπάνω χρήματα στον λογαριασμό του.
    \item[\ref{Τέλος dispute resolved - Τέλος χρημάτων}.4.] Το σύστημα αντιλαμβάνεται ότι προστέθηκαν παραπάνω χρήματα και χρεώνει από τον ενοικιαστή το ποσό που οφείλει, μαζί με 20\% επιτόκιο.
    \item[\ref{Δεν ενημερώνεται η κατοχή}.5.] Η περίπτωση χρήσης συνεχίζεται από το βήμα \ref{Επιστροφή βιβλίου - Τέλος λεφτά δεν φτάνουν} της βασικής ροής.
\end{enumerate}

\subsubsection*{Εναλλακτική Ροή 7: Τέλος χρημάτων ενοικιαστή - Δεν προστίθεται το απαραίτητο ποσό:}
\begin{enumerate}
    \item[\ref{Τέλος dispute resolved - Τέλος χρημάτων}.3.1.] Το σύστημα αντιλαμβάνεται μετά από ένα χρονικό διάστημα ότι ο χρήστης δεν έχει προσθέσει το ποσό που οφείλει στον λογαριασμό του.
    \item[\ref{Τέλος dispute resolved - Τέλος χρημάτων}.3.2.] Το σύστημα προσφέρει αυτόματα το "ποσό ασφαλείας" στον ιδιοκτήτη και απενεργοποιεί τη δυνατότητα του ενοικιαστή να νοικιάσει άλλο βιβλίο, μέχρι να πληρώσει το χρωστούμενο ποσό.
\end{enumerate}

\subsection{Διαχείριση βιβλίων που προσφέρει ο χρήστης προς ενοικίαση}

\end{document}
