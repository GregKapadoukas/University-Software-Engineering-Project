%! TeX program = lualatex
\documentclass[12pt,a4paper]{article}

\usepackage[nil]{babel}
\usepackage{unicode-math}
\usepackage[svgnames]{xcolor}
\usepackage{lmodern}
\usepackage{graphicx}
\usepackage{wrapfig}
\usepackage{float}
\usepackage{parskip}
\usepackage{hyperref}
\usepackage{listings}

\definecolor{codegreen}{rgb}{0,0.6,0}
\definecolor{codegray}{rgb}{0.5,0.5,0.5}
\definecolor{codepurple}{rgb}{0.58,0,0.82}
\definecolor{backcolour}{rgb}{0.95,0.95,0.92}

\lstdefinestyle{mystyle}{
    backgroundcolor=\color{backcolour},   
    commentstyle=\color{codegreen},
    keywordstyle=\color{magenta},
    stringstyle=\color{codepurple},
    basicstyle=\ttfamily\footnotesize,
    breakatwhitespace=false,         
    breaklines=true,                 
    captionpos=b,                    
    keepspaces=true,                 
    showspaces=false,                
    showstringspaces=false,
    showtabs=false,                  
    tabsize=2
}


\lstset{style=mystyle}


\babelprovide[import=el, main, onchar=ids fonts]{greek} % can also do import=el-polyton
\babelprovide[import, onchar=ids fonts]{english}

\babelfont{rm}
          [Language=Default]{Liberation Sans}
\babelfont[english]{rm}
          [Language=Default]{Liberation Sans}
\babelfont{sf}
          [Language=Default]{Liberation Sans}
\babelfont{tt}
          [Language=Default]{Liberation Sans}

%Enter Title Here
 \title{Project-code-v0.1 \\ LibShare}
 \author{\textbf{Ονόματα / ΑΜ / Έτος:} \\ Γρηγόρης Καπαδούκας / 1072484 / 4\textdegree \\ Χρήστος Μπεστητζάνος / 1072615 / 4\textdegree \\ Νικόλαος Αυγέρης / 1067508 / 5\textdegree \\ Περικλής Κοροντζής / 1072563 / 4\textdegree\\ \href{https://github.com/GregKapadoukas/University-Software-Engineering-Project}{\color{blue}GitHub Link}}

\begin{document}

\makeatletter
\begin{center}
	\LARGE{\@title} \\
	\pagebreak
    \begin{LARGE}\@author\end{LARGE}
    \pagebreak
\end{center}

%Insert Body Here
\section{Σύνδεσμος GitHub με Κώδικα}
Ο κώδικας που γράφτηκε για την εργασία βρίσκεται στο repository στον παρακάτω σύνδεσμο, που οδηγεί απευθείας στο κώδικα στο τελευταίο commit πριν την υποβολή του Project-code-v0.1:

\textcolor{blue}{\href{https://github.com/GregKapadoukas/University-Software-Engineering-Project/tree/master/Code}{https://github.com/GregKapadoukas/University-Software-Engineering-\\Project/tree/master/Code}}

\section{Οδηγίες Εκτέλεσης του Κώδικα και Εγκατάσταση Dependencies}
Για την διεκπεραίωση αυτής της εργασίας έχουμε επιλέξει να χρησιμοποιήσουμε γλώσσα προγραμματισμού Python μαζί τη βιβλιοθήκη CustomTKinter για το GUI.

\begin{enumerate}
    \item Αρχικά λοιπόν πρέπει να εγκατασταθεί η Python από την official σελίδα τους ή από κάποιο package manager. Η έκδοση που χρησιμοποιήσαμε για την συγγραφή του κώδικα είναι η 3.11.3.

    \item Έπειτα πρέπει να εγκατασταθεί η βιβλιοθήκη CustomTKinter, μέσω της εξής εντολής σε τερματικό ή cmd:

\begin{lstlisting}[language=Bash]
pip install customtkinter\end{lstlisting}

    \item Έπειτα εκτελούμε τον κώδικα μέσω της main, όπως φαίνεται παρακάτω:

\begin{lstlisting}[language=Bash]
python main.py\end{lstlisting}
\end{enumerate}

\textbf{Εναλλακτικά:}

Γίνεται να χρησιμοποιηθεί το distribution Minicoda για τη δημιουργία ενός Python virtual environment στο οποίο θα εκτελεστεί ο κώδικας. Αυτή είναι η προσέγγιση που προτιμήσαμε και εμείς και προτείνουμε. Τα βήματα είναι τα εξής:

\begin{enumerate}
    \item Εγκατάσταση του Miniconda μέσω του installer στη σελίδα:

        \textcolor{blue}{\href{https://docs.conda.io/en/latest/miniconda.html}{https://docs.conda.io/en/latest/miniconda.html}}
    \item Δημιουργία ενός conda virtual environment που χρησιμοποιεί την έκδοση 3.11.3 της Python που χρησιμοποιούμε και εμείς μέσω της εξής εντολής:

\begin{lstlisting}[language=Bash]
conda create -n ctk python=3.11.3\end{lstlisting}

    \item Ενεργοποίηση του conda environment μέσω της εξής εντολής:

\begin{lstlisting}[language=Bash]
conda activate ctk\end{lstlisting}

    \item Εγκατάσταση του CustomTKinter μέσω της εξής εντολής:

\begin{lstlisting}[language=Bash]
pip install customtkinter\end{lstlisting}

    \item Εκτέλεση του κώδικα μέσω της παρακάτω εντολής:
\begin{lstlisting}[language=Bash]
python main.py\end{lstlisting}

\end{enumerate}

\section{Τι Περιλαμβάνει η Έκδοση v0.1}

Η έκδοση v0.1 του Project-code περιλαμβάνει το σύστημα οδήγησης μεταξύ των βασικών οθονών (navigation bar), με χρήση placeholder οθονών μέχρι αυτές να χτιστούν. Εξαίρεση αποτελεί η οθόνη αναζήτησης, η οποία έχει σχεδιαστεί, αλλά δεν έχει υλοποιηθεί η λειτουργικότητά της.

Ο κώδικας για τις κλάσεις του Domain-model (μετέπειτα Class-diagram) δεν έχει υλοποιηθεί ακόμα και θα υλοποιηθεί σε επόμενο παραδοτέο.

\section{Συμμετοχή και Ρόλοι στη Συγγραφή του Κειμένου}
\begin{enumerate}
	\item \textbf{Γρηγόρης Καπαδούκας:} Συγγραφέας τεχνικού κειμένου, Συγγραφέας όλου του κώδικα για την έκδοση v0.1.
\end{enumerate}

\end{document}
