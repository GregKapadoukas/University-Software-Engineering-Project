%! TeX program = lualatex
\documentclass[12pt,a4paper]{article}

\usepackage[nil]{babel}
\usepackage{unicode-math}
\usepackage[svgnames]{xcolor}
\usepackage{lmodern}
\usepackage{graphicx}
\usepackage{wrapfig}
\usepackage{float}
\usepackage{parskip}

\babelprovide[import=el, main, onchar=ids fonts]{greek} % can also do import=el-polyton
\babelprovide[import, onchar=ids fonts]{english}

\babelfont{rm}
          [Language=Default]{Liberation Sans}
\babelfont[english]{rm}
          [Language=Default]{Liberation Sans}
\babelfont{sf}
          [Language=Default]{Liberation Sans}
\babelfont{tt}
          [Language=Default]{Liberation Sans}

%Enter Title Here
 \title{Project-description v0.1 \\ LibShare}
\author{\textbf{Ονόματα / ΑΜ / Έτος:} \\ Γρηγόρης Καπαδούκας / 1072484 / 4\textdegree \\ Χρήστος Μπεστητζάνος / 1072615 / 4\textdegree \\ Νικόλαος Αυγέρης / 1067508 / 5\textdegree \\ Περικλής Κοροντζής / 1072563 / 4\textdegree}

\begin{document}

\makeatletter
\begin{center}
	\LARGE{\@title} \\
	\pagebreak
	\begin{LARGE}\@author\end{LARGE} \\
\end{center}
\pagebreak

%Insert Body Here
\section{Περιγραφή:}
Το προς δημιουργία project λογισμικού αποτελεί μια εφαρμογή Peer-2-Peer ενοικίασης βιβλίων. Δηλαδή η εφαρμογή αυτή επιτρέπει χρήστες να νοικιάζουν βιβλία ο ένας από τον άλλο για χρηματική ανταμοιβή, με ενσωματωμένη ασφάλεια έναντι κλοπών. Ο χρήστης σε πρώτο βήμα θα μπορεί να δημιουργεί προσωπικό λογαριασμό στην εφαρμογή, εισάγοντας το προσωπικό του email, ένα username και ένα password. Μετά τη δημιουργία του λογαριασμού του ο χρήστης θα μπορεί να εισέρχεται στην εφαρμογή εισάγοντας τα στοιχεία του. Στην εφαρμογή του δίνονται οι εξής δυνατότητες:

\begin{itemize}
	\item Θα μπορεί να νοικιάσει ένα βιβλίο από άλλους χρήστες της εφαρμογής. Κατά τη λειτουργία αυτή θα εμφανίζεται μια λίστα από διαθέσιμα βιβλία προς ενοικίαση από χρήστες κοντά του (στην ίδια πόλη με αυτόν) που κάνουν συναλλαγές πρόσωπο με πρόσωπο ή από χρήστες που μένουν μακριά αλλά δέχονται ταχυδρομικές συναλλαγές. 

Έτσι αφού διαλέξει ο χρήστης το βιβλίο που θέλει, του εμφανίζεται η λίστα των χρηστών που προσφέρουν το βιβλίο αυτό για ενοικίαση. Σε αυτή τη λίστα βλέπει ο χρήστης τους τρόπους συναλλαγής που δέχεται κάθε χρήστης (ταχυδρομικώς ή πρόσωπο με πρόσωπο). Αφού επιλέξει έναν μπορεί να του κάνει "Αίτηση ενοικίασης". Έπειτα της αίτησης, ενημερώνεται ο χρήστης-ιδιοκτήτης του βιβλίου και έχει την επιλογή να δεχτεί ή να απορρίψει την αίτηση αυτή.

Αν τη δεχτεί, οι δύο χρήστες βλέπουν τα στοιχεία επικοινωνίας ο ένας του άλλου και αφού λάβει ο ενοικιαστής το βιβλίο ενημερώνουν οι χρήστες την εφαρμογή και χρεώνεται αυτομάτως ο ενοικιαστής ημερήσια, μέχρι το βιβλίο να επιστραφεί στον ιδιοκτήτη. 

Αλλίως αν η αίτηση του ενοικιαστή απορριφθεί, τότε αυτός ενημερώνεται και δεν γίνεται συναλλαγή.

	\item Ο χρήστης θα μπορεί επίσης να κάνει αναζήτηση άλλων χρηστών με βάση το username τους και να δει τα βιβλία που προσφέρουν για ενοικίαση.

	\item Οι χρήστες θα μπορούν έπειτα από μια συναλλαγή να κάνουν reviews ο ένας στον άλλο με τη μορφή 1 έως 5 αστεριών, και το συνολικό "σκορ" του κάθε χρήστη εμφανίζεται δίπλα στο προφιλ του για όλους τους υπόλοιπους χρήστες.

	\item Ο κάθε χρήστης θα μπορεί να προβάλλει και να επεξεργάζεται τη λίστα των βιβλιών που του ανήκουν και προσφέρει για ενοικίαση.

	\item Θα υπάρχει δυνατότητα για τους χρήστες να προβάλλουν το ιστορικό των συναλλαγών τους με τους άλλους χρήστες, μαζί με την κατάσταση των εκκρεμών συναλλαγών και αιτήσεων.

	\item Θα μπορεί ο κάθε χρήστης να προσθέσει και να αφαιρέσει λεφτά από τον λογαριασμό του. Η αυτόματη μεταφορά των χρημάτων μεταξύ χρηστών κατά τη διάρκεια μιας ενοικίαση γίνεται από τον λογαριασμό αυτό. Επίσης για να επιτευχθεί μια ενοικίαση μεταφέρεται ένα ποσό προκαταβολής από τον ενοικιαστή προς τον ιδιοκτήτη, με σκοπό την προστασία του ιδιοκτήτη από κλοπή.

	\item Ο χρήστης θα μπορεί να προβάλλει και να αλλάζει τα στοιχεία του και τον κωδικό πρόσβασής του.

	\item Θα υπάρχει επίσης λειτουργικότητα για τους χρήστες να δημιουργούν και να προβάλλουν "αίτηματα" για βιβλία που δεν υπάρχουν διαθέσιμα προς ενοικίαση στην πλατφόρμα, με προσφορά επιπλέον ανταμοιβής σε όποιον ιδιοκτήτη του βιβλίου αυτού το προσφέρει στον χρήστη που το ζητάει. Η ανταμοιβή δεσμεύεται μαζί με την προκαταβολή από τον ενοικιαστή κατά την συναλλαγή.

	\item Θα υπάρχει δυνατότητα για προσθήκη άλλων χρηστών στη λίστα "Αγαπημένων", όπου τα βιβλία που τους ανήκουν και προσφέρονται για ενοικίαση εμφανίζονται με προτεραιότητα στη λίστα των διαθέσιμων βιβλίων.

	\item Θα υπάρχει δυνατότητα επικοινωνίας με υποστήριξη πελατών με \\σκοπό την επίλυση τυχόν διαφωνιών πελατών, στην περίπτωση που κάποιος χρήστης νιώθει ότι έχει αδικηθεί, όπως για παράδειγμα αν του έχει σταλθεί ταχυδρομικώς το λάθος βιβλίο ή στη περίπτωση που ένα βιβλίο έχει υποστεί μεγάλη φυσική φθορά από έναν ενοικιαστή. 

		Η υποστήριξη πελατών λειτουργεί με βάση προκαθορισμένης πολιτικής προστασίας πελατών. Έχει σκοπό να τιμωρήσει κακόβουλες συμπεριφορές και έχει τη δυνατότητα να αποδεσμεύσει το ποσό προκαταβολής και να το χρησιμοποιήσει για να αποζημιωθούν οι χρήστες, όταν το ορίζει η πολιτική.

\end{itemize}

\section{Mock-up Screens}

\section{Συμμετοχή και Ρόλοι στη Συγγραφή του Κειμένου}
\begin{enumerate}
	\item \textbf{Γρηγόρης Καπαδούκας:} Author
	\item \textbf{Χρήστος Μπεστητζάνος:} Editor, Contributor
	\item \textbf{Νικόλαος Αυγέρης:} Peer Reviewer
\end{enumerate}

\end{document}
