%! TeX program = lualatex
\documentclass[12pt,a4paper]{article}

\usepackage[nil]{babel}
\usepackage{unicode-math}
\usepackage[svgnames]{xcolor}
\usepackage{lmodern}
\usepackage{graphicx}
\usepackage{wrapfig}
\usepackage{float}
\usepackage{parskip}
\usepackage[font=small,labelfont=bf]{caption}
\setlength{\emergencystretch}{3em}

\babelprovide[import=el, main, onchar=ids fonts]{greek} % can also do import=el-polyton
\babelprovide[import, onchar=ids fonts]{english}

\babelfont{rm}
          [Language=Default]{Liberation Sans}
\babelfont[english]{rm}
          [Language=Default]{Liberation Sans}
\babelfont{sf}
          [Language=Default]{Liberation Sans}
\babelfont{tt}
          [Language=Default]{Liberation Sans}

%Enter Title Here
 \title{Project-description-v0.2 \\ LibShare}
\author{\textbf{Ονόματα / ΑΜ / Έτος:} \\ Γρηγόρης Καπαδούκας / 1072484 / 4\textdegree \\ Χρήστος Μπεστητζάνος / 1072615 / 4\textdegree \\ Νικόλαος Αυγέρης / 1067508 / 5\textdegree \\ Περικλής Κοροντζής / 1072563 / 4\textdegree}

\begin{document}

\makeatletter
\begin{center}
	\LARGE{\@title} \\
	\pagebreak
	\begin{LARGE}\@author\end{LARGE} \\
\end{center}
\pagebreak

%Insert Body Here
\section{Περιγραφή:}
\label{Περιγραφή}
Το προς δημιουργία project λογισμικού αποτελεί μια εφαρμογή Peer-2-Peer ενοικίασης βιβλίων. Δηλαδή η εφαρμογή αυτή επιτρέπει χρήστες να νοικιάζουν βιβλία ο ένας από τον άλλο για χρηματική ανταμοιβή, με ενσωματωμένη ασφάλεια έναντι κλοπών. Ο χρήστης σε πρώτο βήμα θα μπορεί να δημιουργεί προσωπικό λογαριασμό στην εφαρμογή, εισάγοντας το προσωπικό του email, ένα username και ένα password. Μετά τη δημιουργία του λογαριασμού του ο χρήστης θα μπορεί να εισέρχεται στην εφαρμογή εισάγοντας τα στοιχεία του. Στην εφαρμογή του δίνονται οι εξής δυνατότητες:

\subsection{Σύστημα ενοικιάσεων βιβλίων}
Θα μπορεί να νοικιάσει ένα βιβλίο από άλλους χρήστες της εφαρμογής. Κατά τη λειτουργία αυτή θα του εμφανίζεται μια λίστα από διαθέσιμα βιβλία προς ενοικίαση από άλλους χρήστες της εφαρμογής. Θα υπάρχει δυνατότητα για συναλλαγή πρόσωπο με πρόσωπο, αν οι δύο χρήστες μένουν κοντά και μέσω ταχυδρομικής αποστολής, ασχέτως αν οι χρήστες μένουν κοντά ή μακριά.

Έτσι αφού ο χρήστης κάνει αναζήτηση στη πλατφόρμα και βρει το βιβλίο που θέλει, θα του εμφανίζεται η λίστα των χρηστών που προσφέρουν το βιβλίο αυτό για ενοικίαση. Σε αυτή τη λίστα βλέπει ο χρήστης τους τρόπους συναλλαγής που δέχεται κάθε χρήστης (ταχυδρομικώς ή πρόσωπο με πρόσωπο), την τιμή που ζητάει ο καθένας και το προσωπικό του "σκορ" που προκύπτει από reviews των άλλων χρηστών (θα εξηγηθεί περισσότερο παρακάτω). Αφού επιλέξει έναν μπορεί να του κάνει "Αίτηση ενοικίασης". Έπειτα της αίτησης, ενημερώνεται ο χρήστης - ιδιοκτήτης του βιβλίου και έχει την επιλογή να δεχτεί ή να απορρίψει την αίτηση αυτή.

Αν τη δεχτεί, οι δύο χρήστες βλέπουν τα στοιχεία επικοινωνίας ο ένας του άλλου και αφού λάβει ο ενοικιαστής το βιβλίο ενημερώνουν οι χρήστες την εφαρμογή και χρεώνεται αυτομάτως ο ενοικιαστής ημερήσια, μέχρι το βιβλίο να επιστραφεί στον ιδιοκτήτη και να δηλωθεί η επιστροφή στη πλατφόρμα. 
Αλλιώς αν η αίτηση του ενοικιαστή απορριφθεί, τότε o χρήστης - ενοικιαστής ενημερώνεται και δεν γίνεται συναλλαγή.

Επίσης ο κάθε χρήστης θα μπορεί να αναρτήσει τα βιβλία του στη πλατφόρμα, προσφέροντας τη δυνατότητα στους άλλου χρήστες να νοικιάζουν τα βιβλία του, με χρηματική ανταμοιβή που ορίζει ο ίδιος. Ο χρήστης θα μπορεί επίσης να επεξεργάζεται τη λίστα των βιβλίων που του ανήκουν και προσφέρει για ενοικίαση, προσθέτοντας και αφαιρώντας τυχόν βιβλία.

Ακόμα θα υπάρχει δυνατότητα για τους χρήστες να προβάλλουν το ιστορικό των συναλλαγών τους, μαζί με την κατάσταση των εκκρεμών συναλλαγών και αιτήσεων.

Τέλος οι χρήστες θα μπορούν επίσης έπειτα από μια συναλλαγή να κάνουν reviews ο ένας στον άλλο με τη μορφή 1 έως 5 αστεριών, με προαιρετική προσθήκη σχολίου. Έτσι το συνολικό "σκορ" του κάθε χρήστη εμφανίζεται δίπλα στο όνομά του στη λίστα αναζήτησης βιβλίων και στο προφίλ του φαίνεται το σκορ μαζί με τα σχόλια στους υπόλοιπους χρήστες.

\subsection{Αναζήτηση άλλων χρηστών, προσθήκη αγαπημένων χρηστών και σύστημα ενημερώσεων}
Ο χρήστης θα μπορεί επίσης να κάνει αναζήτηση άλλου χρήστη με βάση το username τους και να δει τα στοιχεία τους (username, το σκορ και τα σχόλιά του, την πόλη που μένει, ένα description που ορίζει ο ίδιος) και τα βιβλία που προσφέρει για ενοικίαση.

Επίσης θα υπάρχει δυνατότητα για προσθήκη των άλλων χρηστών στη λίστα "Αγαπημένων", όπου τα βιβλία που τους ανήκουν και προσφέρονται για ενοικίαση εμφανίζονται με προτεραιότητα στη λίστα των διαθέσιμων βιβλίων κατά την αναζήτηση.

Επίσης το σύστημα των αγαπημένων θα συνδέεται με το σύστημα των ενημερώσεων, με το οποίο ο χρήστης ενημερώνεται από το σύστημα όταν ένας χρήστης στη λίστα αγαπημένων του προσθέτει νέο βιβλίο προς ενοικίαση στο σύστημα.

\subsection{Σύστημα αιτήσεων βιβλίων που δεν προσφέρονται ήδη στην πλατφόρμα}
Θα υπάρχει επίσης λειτουργικότητα για τους χρήστες να δημιουργούν και να προβάλλουν "αιτήματα" για βιβλία που δεν υπάρχουν διαθέσιμα προς ενοικίαση στην πλατφόρμα, με προσφορά επιπλέον ανταμοιβής σε όποιον ιδιοκτήτη του βιβλίου αυτού το προσφέρει στον χρήστη που το ζητάει. Με αυτόν τον τρόπο διευκολύνεται η εύρεση βιβλίων που δεν είναι ήδη διαθέσιμα στη πλατφόρμα.

Επίσης θα μπορεί ο χρήστης να αναζητεί, να προβάλλει και να πληρεί αιτήσεις άλλων χρηστών, εφόσον έχει το βιβλίο που ζητείται, και έτσι να κάνει αίτηση εκπλήρωσης αίτησης στον χρήστη που την έφτιαξε και να αρχικοποιηθεί συναλλαγή μαζί του.

\subsection{Σύστημα διαχείρισης λογαριασμών}
Ο χρήστης θα μπορεί να προβάλλει και να αλλάζει τα στοιχεία του, δηλαδή το username, το email το, σύντομο description, τη πόλη κατοικίας, την  εικόνα χρήστη και τον κωδικό πρόσβασής του. Θα μπορεί επίσης να προσθέσει και να αφαιρέσει λεφτά από τον λογαριασμό του. 

\subsection{Σύστημα προστασίας χρηστών από κλοπές και φθορές}
Για την προστασία των ιδιοκτητών από κλοπές, θα γίνεται αυτόματη μεταφορά των χρημάτων μεταξύ χρηστών κατά τη διάρκεια μιας ενοικίαση. 

Επίσης για να επιτευχθεί μια ενοικίαση δεσμεύεται ένα "ποσό ασφαλείας" από τον ενοικιαστή, με σκοπό την προστασία του ιδιοκτήτη από κλοπή ή φθορά του βιβλίου. Στην περίπτωση που η συναλλαγή ολοκληρωθεί χωρίς κανένα θέμα το ποσό αυτό ασφαλείας θα επιστρέφεται στον ενοικιαστή αυτομάτως.

Στην περίπτωση όμως διαμάχης μεταξύ ενοικιαστή και ιδιόκτητη θα υπάρχει δυνατότητα επικοινωνίας με υποστήριξη πελατών με σκοπό την επίλυση των διαφωνιών όπου είναι εφικτό. Παραδείγματα διαμαχών μπορεί να είναι η κλοπή ή φθορά του βιβλίου, όπου ο ιδιοκτήτης αποζημιώνεται μέσω του "ποσού ασφαλείας" ή θα μπορούσε να είναι η αποστολή ταχυδρομικώς του λάθους βιβλίου, όπου ο ενοικιαστής δεν χρειάζεται να πληρώσει τίποτα, και του αποδεσμεύεται αμέσως το "ποσό ασφαλείας". Και στις δύο περιπτώσεις το άτομο με κακή τιμωρείται από την πλατφόρμα με περιορισμούς στις ενοικιάσεις, και από τους ίδιους τους χρήστες μέσω των κακών σκορ.

Η υποστήριξη πελατών σημειώνουμε λειτουργεί με βάση προκαθορισμένης πολιτικής προστασίας πελατών. Έχει σκοπό να τιμωρήσει κακόβουλες συμπεριφορές και έχει τη δυνατότητα να αποδεσμεύσει το ποσό προκαταβολής και να το χρησιμοποιήσει για να αποζημιωθούν οι χρήστες, μόνο όταν το ορίζει η πολιτική.

\section{Mock-up Οθόνες}

\textbf{Σημέιωση:} Οι οθόνες στο κεφάλαιο αυτό είναι ενδεικτικές και δεν θα αποτελούν αναγκαστικά την ακριβή δομή του τελικού project. Λειτουργούν παραπάνω για ένδειξη του περιεχομένου της τελικής οθόνης και όροι όπως "στο κάτω μέρος της οθόνης" αναφέρονται μόνο στη δομή του σχήματος που δίνεται και όχι αναγκαστικά του τελικού project.

\subsection{Rent Book:}

Στο Σχήμα \ref{Οθόνη "Rent Book" λειτουργικότητας} παρουσιάζεται mock-up οθόνη της λειτουργικότητας της ενοικίασης βιβλίων.

\begin{figure}[H]
	\makebox[\textwidth]{\includegraphics[width=\textwidth]{Mockup Screens/Rent Book.png}}
	\caption{Οθόνη "Rent Book" λειτουργικότητας}
	\label{Οθόνη "Rent Book" λειτουργικότητας}
\end{figure}

Η οθόνη αυτή περιέχει έναν πίνακα με στήλες με τίτλους "Title", "Author", "Owner", "Delivery Way" και ένα κουμπί "Request Rental". 

Η λειτουργικότητα του "Request Rental" είναι αυτή που εξηγείται ως "αίτηση ενοικίασης" στο κεφάλαιο \ref{Περιγραφή}.

Τα κελιά του πίνακα στη τελική έκδοση θα περιέχουν όλα τα διαθέσιμα βιβλία και τις πληροφορίες τους ανά στήλη, μαζί με την τιμή ενοικίασης ανά ημέρα (κάτω από το "Request Rental").

Επίσης υπάρχει και λειτουργία αναζήτησης συγκεκριμένου βιβλίου στο κάτω μέρος της οθόνης.

\subsection{User Search}

Στο Σχήμα \ref{Οθόνη "User Search" λειτουργικότητας} παρουσιάζεται mock-up οθόνη της λειτουργικότητας της αναζήτησης χρηστών.

\begin{figure}[H]
	\makebox[\textwidth]{\includegraphics[width=\textwidth]{Mockup Screens/User Search.png}}
	\caption{Οθόνη "User Search" λειτουργικότητας}
	\label{Οθόνη "User Search" λειτουργικότητας}
\end{figure}

Παρατηρούμε ότι η οθόνη αυτή περιέχει έναν πίνακα με στήλες με τίτλους "Name", "Address", "Num of Books" "Delivery Way" και ένα κουμπί "View Books". 

Το κουμπί "View Books" μεταβιβάζει τον χρήστη σε μια σελίδα όπως τη σελίδα του "Rent Book" (Σχήμα \ref{Οθόνη "Rent Book" λειτουργικότητας} που θα περιέχει μόνο τα βιβλία του χρήστη που επιλέχθηκε.

Τα κελιά του πίνακα στη τελική έκδοση θα περιέχουν όλους τους χρήστες που πληρούν την αναζήτηση και τις πληροφορίες τους ανά στήλη.

Επίσης υπάρχει και λειτουργία αναζήτησης συγκεκριμένου χρήστη στο κάτω μέρος της οθόνης.

\subsection{Review Peers}

Στο Σχήμα \ref{Οθόνη "Review peers" λειτουργικότητας} παρουσιάζεται mock-up οθόνη της λειτουργικότητας του Review άλλων χρηστών.

\begin{figure}[H]
	\makebox[\textwidth]{\includegraphics[width=\textwidth]{Mockup Screens/Review Peers.png}}
	\caption{Οθόνη "Review peers" λειτουργικότητας}
	\label{Οθόνη "Review peers" λειτουργικότητας}
\end{figure}

Παρατηρούμε ότι η οθόνη αυτή περιέχει έναν πίνακα με στήλες με τα ονόματα των χρηστών με τους οποίους έχουν γίνει προηγούμενες συναλλαγές, και το βιβλίο που ενοικιάστηκε.

Επίσης φαίνεται και ένα γραφικό δίπλα σε κάθε συναλλαγή με λειτουργικότητα της κριτικής του άλλου χρήστη από 1 έως 5 "αστεράκια".

Τα κελιά του πίνακα στη τελική έκδοση θα περιέχουν όλους τους χρήστες και τα βιβλία για τις προηγούμενες συναλλαγές που έχουν γίνει.

\subsection{My Listings}

Στο Σχήμα \ref{Οθόνη "My Listings" λειτουργικότητας} παρουσιάζεται mock-up οθόνη της λειτουργικότητας της προβολής των βιβλίων που προσφέρει ο χρήστης σε άλλους να νοικιάσουν, μαζί με δυνατότητα προσθήκης και αφαίρεσης άλλων βιβλίων.

\begin{figure}[H]
	\makebox[\textwidth]{\includegraphics[width=\textwidth]{Mockup Screens/My Listings.png}}
	\caption{Οθόνη "My Listings" λειτουργικότητας}
	\label{Οθόνη "My Listings" λειτουργικότητας}
\end{figure}

Παρατηρούμε ότι η οθόνη αυτή περιέχει έναν πίνακα με στήλες με τίτλους "Title", "Author", "Owner", "Delivery Way" και κουμπιά "Remove Listing" και στο τέλος της σελίδας ένα κουμπί "Add Listing", όπου πλέον ο χρήστης συμπληρώνει τα στοιχεία του βιβλίου που θα νοικιάσει σε ένα popup screen.

Τα κελιά του πίνακα στη τελική έκδοση θα περιέχουν το σύνολο των βιβλίων που προσφέρει ο συνδεδεμένος χρήστης για ενοικίαση από άλλους.

\subsection{Requests}

Στο Σχήμα \ref{Οθόνη "Requests" λειτουργικότητας} παρουσιάζεται mock-up οθόνη της λειτουργικότητας της προβολής αιτήσεων άλλων και της αποδοχή τους (δηλαδή προσφοράς του βιβλίου στον χρήστη που το ζητάει και έκανε την αίτηση και αποδοχή του χρηματικού αντίτιμου). Επίσης στο κάτω μέρος της σελίδας υπάρχει ένα κουμπί που προσφέρει λειτουργικότητα δημιουργίας νέας αίτησης.

\begin{figure}[H]
	\makebox[\textwidth]{\includegraphics[width=\textwidth]{Mockup Screens/Request.png}}
	\caption{Οθόνη "Requests" λειτουργικότητας}
	\label{Οθόνη "Requests" λειτουργικότητας}
\end{figure}

Παρατηρούμε ότι η οθόνη αυτή περιέχει έναν πίνακα με στήλες με τίτλους "Title", "Author", "Owner", "Delivery Way", "User", "Bounty" και κουμπιά "Claim Request" και στο τέλος της σελίδας το κουμπί "Add Requests", όπου πλέον ο χρήστης συμπληρώνει τα στοιχεία του βιβλίου που ζητάει να νοικιάσει σε ένα popup screen.

Τα κελιά του πίνακα στη τελική έκδοση θα περιέχουν το σύνολο των στοιχείων για τα requests που είναι ενεργά εκείνη τη χρονική στιγμή.

\subsection{Transaction History}

Στο Σχήμα \ref{Οθόνη "Transaction History" λειτουργικότητας} παρουσιάζεται mock-up οθόνη της λειτουργικότητας της προβολής συναλλαγών (προηγούμενων και τρεχόντων) καθώς και της αποδοχής ή απόρριψης αιτημάτων και ανανέωσης κατάστασης κατοχής βιβλίου (δηλαδή ανανεώνεται όταν έχει φτάσει το βιβλίο στον ενοικιαστή και πίσω στον ιδιοκτήτη).

\begin{figure}[H]
	\makebox[\textwidth]{\includegraphics[width=\textwidth]{Mockup Screens/Transaction History.png}}
	\caption{Οθόνη "Transaction History" λειτουργικότητας}
	\label{Οθόνη "Transaction History" λειτουργικότητας}
\end{figure}

Παρατηρούμε ότι η οθόνη αυτή περιέχει έναν πίνακα με στήλες με τίτλους "Title", "Author", "Owner", "Delivery Way", "Transaction Status" και εναλλακτικά κουμπιά "Accept Request" και "Deny Request" ή "Toggle Possession Status" ή κανένα κουμπί, αντίστοιχα με την κατάσταση της συναλλαγής, όπως αναφέρεται στο κεφάλαιο \ref{Περιγραφή}.

Τα κελιά του πίνακα στη τελική έκδοση θα περιέχουν όλο το ιστορικό των συναλλαγών για τον χρήστη που είναι συνδεδεμένος.

\subsection{Profile}

Στο Σχήμα \ref{Οθόνη "Profile" λειτουργικότητας} παρουσιάζεται mock-up οθόνη της λειτουργικότητας της προβολής και αλλαγής των στοιχείων του λογαριασμού του συνδεδεμένου χρήστη.

\begin{figure}[H]
	\makebox[\textwidth]{\includegraphics[width=\textwidth]{Mockup Screens/Profile.png}}
	\caption{Οθόνη "Profile" λειτουργικότητας}
	\label{Οθόνη "Profile" λειτουργικότητας}
\end{figure}

Παρατηρούμε ότι η οθόνη αυτή περιέχει έναν πίνακα με γραμμές με τίτλους "Username", "Email", "Address" και "Phone". Στη δεξιά μεριά του πίνακα προβάλλονται κάθε φορά τα στοιχεία του χρήστη.

Στη κάτω μεριά της οθόνης υπάρχει ένα κουμπί "Edit" το οποίο επιτρέπει στον χρήστη να ανανεώσει τα στοιχεία του.

\subsection{Balance}

Στο Σχήμα \ref{Οθόνη "Balance" λειτουργικότητας} παρουσιάζεται mock-up οθόνη της λειτουργικότητας της προβολής χρημάτων στον λογαριασμό και της προσθήκης και αφαίρεσης χρηματικού ποσού από τον λογαριασμό.

\begin{figure}[H]
	\makebox[\textwidth]{\includegraphics[width=\textwidth]{Mockup Screens/Balance.png}}
	\caption{Οθόνη "Balance" λειτουργικότητας}
	\label{Οθόνη "Balance" λειτουργικότητας}
\end{figure}

Παρατηρούμε ότι η οθόνη αυτή περιέχει κουμπιά "Add Money" και "Remove Balance" για την προσθήκη και αφαίρεση χρηματικού ποσού από τον λογαριασμό αντίστοιχα.

Στη μέση της οθόνης υπάρχει ένα πεδίο "Balance" στο οποίο φαίνεται το ποσό που έχει διαθέσιμο ο χρήστης στον λογαριασμό του (έχουν αφαιρεθεί οι προκαταβολές και οι ανταμοιβές για αιτήσεις, τα οποία όμως μπορούν να επιστραφούν στο μέλλον).

\section{Συμμετοχή και Ρόλοι στη Συγγραφή του Κειμένου}

\begin{enumerate}
	\item \textbf{Γρηγόρης Καπαδούκας:} Author
	\item \textbf{Χρήστος Μπεστητζάνος:} Editor, Contributor
	\item \textbf{Νικόλαος Αυγέρης:} Peer Reviewer
	\item \textbf{Περικλής Κοροντζής:} Peer Reviewer
\end{enumerate}

\section{Αλλαγές από έκδοση σε έκδοση}

\subsection{Από έκδοση v0.1 σε έκδοση v0.2}
\subsubsection{Αλλαγές στη περιγραφή}
\begin{itemize}
    \item Αναδόμηση του κειμένου της περιγραφής ώστε να είναι χωρισμένο σε τμήματα και πιο ευανάγνωστο. 
    \item Προσθήκη του συστήματος των ειδοποιήσεων.
    \item Προσθήκη παραπάνω διευκρινήσεων σχετικά με τη λειτουργικότητα των αιτήσεων και των αναζητήσεων.
\end{itemize}

\end{document}
