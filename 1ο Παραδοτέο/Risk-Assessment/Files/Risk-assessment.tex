%! TeX program = lualatex
\documentclass[12pt,a4paper]{article}

\usepackage[nil]{babel}
\usepackage{unicode-math}
\usepackage[svgnames]{xcolor}
\usepackage{lmodern}
\usepackage{graphicx}
\usepackage{wrapfig}
\usepackage{float}
\usepackage{parskip}

\babelprovide[import=el, main, onchar=ids fonts]{greek} % can also do import=el-polyton
\babelprovide[import, onchar=ids fonts]{english}

\babelfont{rm}
          [Language=Default]{Liberation Sans}
\babelfont[english]{rm}
          [Language=Default]{Liberation Sans}
\babelfont{sf}
          [Language=Default]{Liberation Sans}
\babelfont{tt}
          [Language=Default]{Liberation Sans}

%Enter Title Here
 \title{Risk-assessment v0.1\\ LibShare}
\author{\textbf{Ονόματα / ΑΜ / Έτος:} \\ Γρηγόρης Καπαδούκας / 1072484 / 4\textdegree \\ Χρήστος Μπεστητζάνος / 1072615 / 4\textdegree \\ Νικόλαος Αυγέρης / 1067508 / 5\textdegree \\ Περικλής Κοροντζής / 1072563 / 4\textdegree}

\begin{document}

\makeatletter
\begin{center}
	\LARGE{\@title} \\
	\pagebreak
\end{center}
\begin{LARGE}\@author\end{LARGE} \\
\pagebreak

%Insert Body Here
\section{Ανάλυση Κινδύνων Έργου}

\textbf{Σημείωση:} Για την ανάλυση ρίσκων χρησιμοποιήσαμε τα πεδία που θεωρήσαμε ότι εφαρμόζουν από τη φόρμα καταγραφής κινδύνων στις διαφάνειες του μαθήματος. Όταν η τεχνική αντιμετώπισης και μετριασμού συμπίπτουν τις έχουμε βάλει μάζι.

Επίσης στο πεδίο της συνδεόμενης δραστηριότητας αναφερόμαστε στις δραστηριότητες του Σχήματος 4 στο Project-plan v0.1.pdf.

Τέλος θεωρούμε, όπως αναφέραμε και στο Project-plan v0.1.pdf ότι έχουμε θεωρήσει ότι κάνουμε ίδρυση startup εταιρίας με σκοπό την διεκπεραίωση του έργου.

\subsection{Κίνδυνος: Καθυστέρηση Συγγραφής Κώδικα}
\begin{itemize}
	\item \textbf{Πιθανότητα Κινδύνου:} Μεσαία
	\item \textbf{Συνδεόμενη Δραστηριότητα:} TY 2, TY 14 - TY 22
	\item \textbf{Υπεύθυνος Αντιμετώπισης:} Όλη η ομάδα
	\item \textbf{Σοβαρότητα:} Υψηλό
	\item \textbf{Σοβαρότητα Συνεπειών:} Μεσαίο
	\item \textbf{Τύπος:} Σχέδιο
	\item \textbf{Στρατηγική Μετριασμού Κινδύνου:} Προσαρμοστικότητα εργασίας
	\item \textbf{Στρατηγική Αντιμετώπισης Κινδύνου:} Κατάλληλη προετοιμασία, ορισμός εφικτών στόχων σε συγκεκριμένα χρονικά διαστήματα
	\item \textbf{Κριτήρια Απενεργοποίησης Κινδύνου:} Επιστροφή στα όρια του χρονοδιαγράμματος
\end{itemize}

\subsection{Κίνδυνος: Έλλειψη Χρηματοδότησης}
\begin{itemize}
	\item \textbf{Πιθανότητα Κινδύνου:} Υψηλή
	\item \textbf{Συνδεόμενη Δραστηριότητα:} TY 2
	\item \textbf{Υπεύθυνος Αντιμετώπισης:} Όλη η ομάδα
	\item \textbf{Σοβαρότητα:} Υψηλή
	\item \textbf{Σοβαρότητα Συνεπειών:} Υψηλή
	\item \textbf{Τύπος:} Σχέδιο, Κόστος
	\item \textbf{Στρατηγική Μετριασμού Κινδύνου:} Περιορισμός μη αναγκαίων εξόδων
	\item \textbf{Στρατηγική Αντιμετώπισης Κινδύνου:} Σωστή παρουσίαση του \\σκεπτικού μας, επιλογή πρωτότυπου θέματος με διαθέσιμο χώρο στην αγορά
	\item \textbf{Κριτήρια Απενεργοποίησης Κινδύνου:} Εύρεση χρηματοδότησης
\end{itemize}

\subsection{Κίνδυνος: Μη Επαρκής Νομική Κάλυψη}
\begin{itemize}
	\item \textbf{Πιθανότητα Κινδύνου:} Χαμηλή
	\item \textbf{Συνδεόμενη Δραστηριότητα:} TY 4
	\item \textbf{Υπεύθυνος Αντιμετώπισης:} Όλη η ομάδα
	\item \textbf{Σοβαρότητα:} Υψηλή
	\item \textbf{Σοβαρότητα Συνεπειών:} Υψηλή
	\item \textbf{Τύπος:} Σχέδιο
	\item \textbf{Στρατηγική Μετριασμού - Αντιμετώπισης Κινδύνου:} Σωστή νομική συμβουλή
	\item \textbf{Κριτήρια Απενεργοποίησης Κινδύνου:} Διαβεβαίωση ειδικού
\end{itemize}

\subsection{Κίνδυνος: Αστοχία Σχεδιασμού}
\begin{itemize}
	\item \textbf{Πιθανότητα Κινδύνου:} Μικρή
	\item \textbf{Συνδεόμενη Δραστηριότητα:} ΤΥ 2 - TY 4
	\item \textbf{Υπεύθυνος Αντιμετώπισης:} Project manager
	\item \textbf{Σοβαρότητα:} Υψηλή
	\item \textbf{Σοβαρότητα Συνεπειών:} Μεσαία
	\item \textbf{Τύπος:} Σχέδιο
	\item \textbf{Στρατηγική Μετριασμού - Αντιμετώπισης Κινδύνου:} Συνεχείς φάσεις ελέγχου και επανασχεδιασμού ανάμεσα στη συγγραφή κώδικα 
	\item \textbf{Κριτήρια Απενεργοποίησης Κινδύνου:} Επιστροφή στο χρονοδιάγραμμα
\end{itemize}

\subsection{Κίνδυνος: Ασυνεννοησία με τον πελάτη / χρηματοδότη}
\begin{itemize}
	\item \textbf{Πιθανότητα Κινδύνου:} Υψηλή
	\item \textbf{Συνδεόμενη Δραστηριότητα:} ΤΥ 1
	\item \textbf{Υπεύθυνος Αντιμετώπισης:} Project manager
	\item \textbf{Σοβαρότητα:} Μεσαία
	\item \textbf{Σοβαρότητα Συνεπειών:} Υψηλή
	\item \textbf{Τύπος:} Σχέδιο, Κώδικας
	\item \textbf{Στρατηγική Μετριασμού Κινδύνου:} Συνεχείς φάσεις ελέγχου και επανασχεδιασμού ανάμεσα στη συγγραφή κώδικα
	\item \textbf{Στρατηγική Αντιμετώπισης Κινδύνου:} Ξεκάθαρη επικοινωνία και καθορισμός απαιτήσεων πριν την έναρξη συγγραφής
	\item \textbf{Κριτήρια Απενεργοποίησης Κινδύνου:} Σύμπλευση απόψεων
\end{itemize}

\subsection{Κίνδυνος: Άστοχο Marketing}
\begin{itemize}
	\item \textbf{Πιθανότητα Κινδύνου:} Μεσαία
	\item \textbf{Συνδεόμενη Δραστηριότητα:} ΤΥ 1
	\item \textbf{Υπεύθυνος Αντιμετώπισης:} Project manager
	\item \textbf{Σοβαρότητα:} Μεσαία
	\item \textbf{Σοβαρότητα Συνεπειών:} Μεσαία
	\item \textbf{Τύπος:} Σχέδιο
	\item \textbf{Στρατηγική Μετριασμού - Αντιμετώπισης Κινδύνου:} Επικοινωνία με ειδικό
	\item \textbf{Κριτήρια Απενεργοποίησης Κινδύνου:} Διαβεβαίωση ειδικού
\end{itemize}

\subsection{Κίνδυνος: Προβλήματα Ασφάλειας}
\begin{itemize}
	\item \textbf{Πιθανότητα Κινδύνου:} Χαμηλή
	\item \textbf{Συνδεόμενη Δραστηριότητα:} ΤΥ 24
	\item \textbf{Υπεύθυνος Αντιμετώπισης:} Υπεύθυνοι ασφαλείας (Σχήμα 2 στο Project-plan v0.1.pdf)
	\item \textbf{Σοβαρότητα:} Υψηλή
	\item \textbf{Σοβαρότητα Συνεπειών:} Υψηλή
	\item \textbf{Τύπος:} Ποιότητα
	\item \textbf{Στρατηγική Μετριασμού - Αντιμετώπισης Κινδύνου:} Συνεχόμενα τεστ και δοκιμές ως προς την ασφάλεια του προϊόντος.
	\item \textbf{Κριτήρια Απενεργοποίησης Κινδύνου:} Επιτυχής επιδιόρθωση \\προβλημάτων ασφάλειας που τυχόν προκύψουν κάθε φορά (ποτέ δεν εξαλείφεται τελείως ο κίνδυνος)
\end{itemize}

\subsection{Κίνδυνος: Δυσλειτουργίες Κώδικα}
\begin{itemize}
	\item \textbf{Πιθανότητα Κινδύνου:} Μεσαία
	\item \textbf{Συνδεόμενη Δραστηριότητα:} TY 14 - TY 22
	\item \textbf{Υπεύθυνος Αντιμετώπισης:} Όλη η ομάδα
	\item \textbf{Σοβαρότητα:} Υψηλή
	\item \textbf{Σοβαρότητα Συνεπειών:} Υψηλή
	\item \textbf{Τύπος:} Ποιότητα
	\item \textbf{Στρατηγική Μετριασμού - Αντιμετώπισης Κινδύνου:} Χρήση σωστών πρότυπων συγγραφής κώδικα
	\item \textbf{Κριτήρια Απενεργοποίησης Κινδύνου:} Επιτυχής / σωστή λειτουργία κώδικα
\end{itemize}

\subsection{Κίνδυνος: Λάθος Χρησιμοποίηση Κεφαλαίου}
\begin{itemize}
	\item \textbf{Πιθανότητα Κινδύνου:} Χαμηλή
	\item \textbf{Συνδεόμενη Δραστηριότητα:} Σύνολο έργου
	\item \textbf{Υπεύθυνος Αντιμετώπισης:} Project manager
	\item \textbf{Σοβαρότητα:} Υψηλή
	\item \textbf{Σοβαρότητα Συνεπειών:} Υψηλή
	\item \textbf{Τύπος:} Κόστος, οργάνωση
	\item \textbf{Στρατηγική Μετριασμού - Αντιμετώπισης Κινδύνου:} Σωστή ανάλυση κόστους και προγραμματισμού εξόδων 
	\item \textbf{Κριτήρια Απενεργοποίησης Κινδύνου:} Επιστροφή στα όρια του προϋπολογισμού
\end{itemize}

\section{Συμμετοχή και Ρόλοι στη Συγγραφή του Κειμένου}
\begin{enumerate}
	\item \textbf{Χρήστος Μπεστητζάνος:} Author
	\item \textbf{Γρηγόρης Καπαδούκας:} Peer Reviewer
\end{enumerate}

\end{document}
