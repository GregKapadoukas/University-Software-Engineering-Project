%! TeX program = lualatex
\documentclass[12pt,a4paper]{article}

\usepackage[nil]{babel}
\usepackage{unicode-math}
\usepackage[svgnames]{xcolor}
\usepackage{lmodern}
\usepackage{graphicx}
\usepackage{wrapfig}
\usepackage{float}
\usepackage{parskip}

\babelprovide[import=el, main, onchar=ids fonts]{greek} % can also do import=el-polyton
\babelprovide[import, onchar=ids fonts]{english}

\babelfont{rm}
          [Language=Default]{Liberation Sans}
\babelfont[english]{rm}
          [Language=Default]{Liberation Sans}
\babelfont{sf}
          [Language=Default]{Liberation Sans}
\babelfont{tt}
          [Language=Default]{Liberation Sans}

%Enter Title Here
 \title{Feasibility-study-v0.1 \\ LibShare}
\author{\textbf{Ονόματα / ΑΜ / Έτος:} \\ Γρηγόρης Καπαδούκας / 1072484 / 4\textdegree \\ Χρήστος Μπεστητζάνος / 1072615 / 4\textdegree \\ Νικόλαος Αυγέρης / 1067508 / 5\textdegree \\ Περικλής Κοροντζής / 1072563 / 4\textdegree}

\begin{document}

\makeatletter
\begin{center}
	\LARGE{\@title} \\
	\pagebreak
\end{center}
\begin{LARGE}\@author\end{LARGE} \\
\pagebreak

%Insert Body Here
\section{Ανάλυση Εφικτότητας Έργου}

Η παρακάτω μελέτη θα εστιάσει στην απάντηση 3ων ερωτήσεων οπού η απαντήσεις των ερωτήσεων τελικά θα απαντήσουν στο ερώτημα:
“Το project που θα επιχειρηθεί είναι εφικτό εντός των περιορισμών που έχουν οριστεί;”

\begin{enumerate}
	\item Το σύστημα που θα δημιουργηθεί συνάδει με τους στόχους της ομάδας μας;
	\item Μπορεί το σύστημα αυτό να δημιουργηθεί εντός  χρονικών και οικονομικών περιορισμών με χρήση ήδη υπαρχουσών τεχνολογιών;
	\item Η δομή του προγράμματος επιτρέπει την ομαλή “συνεργασία” του με αλλά συστήματα τα οποία θα του αυξήσουν την λειτουργικότητα;
\end{enumerate}


\subsection{Το σύστημα που θα δημιουργηθεί συνάδει με τους στόχους της ομάδας μας;}

Το πρόγραμμα αυτό έχει αρκετά στοιχεία και θα γίνει για την ολοκλήρωση του χρήση αρκετών τεχνολογιών οπού τα μέλη θεωρούν ενδιαφέροντα και στοχεύουν να ασχοληθούν και στο μέλλον τους. Είναι ένα έργο milestone για το κάθε μέλος.

\subsection{Μπορεί το σύστημα αυτό να δημιουργηθεί εντός  χρονικών και οικονομικών περιορισμών με χρήση ήδη υπαρχουσών τεχνολογιών;}

Το  ερώτημα θα απαντηθεί με αφαιρετική χρήση των tasks που έχουν οριστεί στο Project-plan δημιουργώντας ένα απλουστευμένο διάγραμμα εξαρτήσεων (Τ2 εξαρτάται από Τ1, Τ3 από Τ2,Τ1 κλπ).

\begin{itemize}
	\item \textbf{Τ1: εύρεση επενδύσεων:}

	Η εύρεση επενδύσεων και οικονομικής υποστήριξης είναι βασικότατο κομμάτι του έργου. Πιστεύουμε ότι η επίτευξη των οικονομικών μας στόχων θα επιτευχθεί με σχετική ευκολία λόγο της ανυπαρξίας παρομοίων προϊόντων στην αγορά καθώς και τις δοκιμασμένης τακτικής χρεώσεις των χρηστών μας.

	\item \textbf{Τ2: Προετοιμασία του προσωπικού:}

	Το προσωπικό είναι άκρας σημασίας πιστεύουμε για οποιοδήποτε έργο οπότε είναι φυσικό να γίνει εστίαση σε αυτόν τον τομέα. Εντούτοις δεν πιστεύουμε ότι θα αντιμετωπίσουμε εξαιρετικά προβλήματα όσον αφορά το προσωπικό κυρίως λόγο των 4ων βασικών μέλλων της ομάδας τα οποία είναι άρτια καταρτισμένα στις τεχνολογίες που θα εφαρμοστούν στο παρόν έργο.

	\item \textbf{Τ3α: Υλοποίηση Interface:}

	Από την φύση του το interface είναι εξαιρετικά δύσκολο να προβλεφθεί με ακρίβεια ο φόρτος εργασίας που θα επιφέρει. Τα βασικά χαρακτηριστικά του είναι αρκετά απλά στον σχεδιασμό και την εκτέλεση τους. Όταν όμως υπάρχει ως στόχος η βελτιστοποίηση του (απλότητα, εμφάνιση, κλπ.) τότε ο φόρτος εργασίας για την ολοκλήρωση του είναι απροσδιόριστος. Βασικό μέλημα της ομάδας μας θα είναι να υλοποιηθεί ένα λειτουργικό (αν και όχι τέλειο) interface μέσα στους χρονικούς περιορισμούς και η εν συνέχεια βελτιστοποίηση του όταν και εάν βρεθεί αρκετός χρόνος.

	\item \textbf{Τ3β: Υλοποίηση Βάσης Δεδομένων:}

	Το σχήμα της βάσης για το παρών έργο είναι αρκετά απλό. Κύρια οντότητα είναι ο “χρήστης” και δευτερεύουσα το “βιβλίο” με ότι άλλο προστεθεί στη συνέχεια συμπληρωματικό με σκοπό την αύξηση των δυνατοτήτων της εφαρμογής η της απόδοσης της. Εντούτοις πιστεύεται ότι η υλοποίηση της θα είναι αρκετά εύκολη και γρήγορη.

	\item \textbf{Τ4: Υλοποίηση λειτουργιών}

	Οι λειτουργίες που θα υλοποιήσουμε αν και περίπλοκες και αχανής σε πρώτη ανάγνωση ουσιαστικά όμως βασίζονται στην απλότητα και ευχρηστία του interface και βάσης δεδομένων. Μια προσεκτική μελέτη των λειτουργιών που θα προσφέρουμε θα δείξει ότι στην ουσία τους είναι μεταβολές της βάσης με μικρό πρότερο φόρτο επεξεργασίας.

	\item \textbf{Τ5: Testing:}

	Η διαδικασία της αποσφαλμάτωσης της εφαρμογής εξαρτάται κυρίως από τα “μονοπάτια επιλογής” του χρήστη, δηλαδή πόση ελευθέρια θα έχει ο χρήστης και πόσες διαφορετικές δράσεις θα μπορεί να εκτελέσει. Η ελευθέρια του χρήστη αν και στοχεύουμε να είναι μεγάλη έχουμε ως γενικό κανόνα την χρήση modular τεχνικών με σκοπό την ψευδαίσθηση διαφορετικών δράσεων από τον χρήστη ενώ στην ουσία ο κώδικας από “πίσω” θα είναι ίδιος αρά λίγος και κατά συνέπεια εύκολος στον έλεγχος.

	\item \textbf{Τ6: Hosting:}

	Στην εποχή του cloud το hosting μιας εφαρμογής έχει “λυθεί” ως πρόβλημα. Με μόνη ανησυχία να παραμένει το κόστος. Το κόστος αυτό εξαρτάται από το “βάρος” της εφαρμογής το οποίο η modular προσέγγισή μας στοχεύει να το κρατήσει χαμηλό. Επιπλέον όπως αναφέρθηκε και προηγουμένως είμαστε αισιόδοξοι όσων αφορά την ανεύρεση επενδυτών γεγονός που μετριάζει κι άλλο το κόστος του hosting.

	\item \textbf{T7: Security:}

	Η λογική μας στο εξής ζήτημα είναι πρόληψη και όχι θεραπεία.H χρήση ήδη αναγνωρισμένων cloud providers είναι ένα σημαντικό βήμα για την ασφάλεια της εφαρμογής μας σε συνδυασμό με την χρήση up to date τεχνικών και τεχνολογιών θα εξασφαλίσει ότι η τελική εφαρμογή θα είναι ήδη αρκετά ασφαλής και θα απαιτεί ελάχιστες τροποποιήσεις μετα τους τελικούς ελέγχους.
\end{itemize}

\subsection{Η δομή του προγράμματος επιτρέπει την ομαλή “συνεργασία” του με αλλά συστήματα τα οποία θα του αυξήσουν την λειτουργικότητα;}

Η απλότητα των υποκείμενων δομών του συστήματος καθώς και η σύγχρονες και ευρέος αποδέκτες τεχνολογίες που θα χρησιμοποιήσουμε καθιστούν το πρόγραμμα αυτό ιδανικό για συγχώνευση και καλή “συνεργασία” με πληθώρα συστημάτων.

Καταλήγοντας θεωρούμε ότι το πρόγραμμα που έχουμε σχεδιάσει και θα υλοποιήσουμε θα τηρήσει τους περιορισμούς που του έχουν τεθεί όσον αφορά το κόστος τους (οικονομικό , χρονικό, κλπ.). Είναι καλοσχεδιασμένο με προσοχή όχι μόνο στην άρτια λειτουργία του αλλά και στην ευκολία διεκπεραίωσης του.

\section{Συμμετοχή και Ρόλοι στη Συγγραφή του Κειμένου}
\begin{enumerate}
	\item \textbf{Περικλής Κοροντζής:} Author
	\item \textbf{Γρηγόρης Καπαδούκας:} Peer Reviewer
	\item \textbf{Χρήστος Μπεστητζάνος:} Peer Reviewer
\end{enumerate}

\end{document}
