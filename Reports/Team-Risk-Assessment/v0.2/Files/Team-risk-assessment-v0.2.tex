%! TeX program = lualatex
\documentclass[12pt,a4paper]{article}

\usepackage[nil]{babel}
\usepackage{unicode-math}
\usepackage[svgnames]{xcolor}
\usepackage{lmodern}
\usepackage{graphicx}
\usepackage{wrapfig}
\usepackage{float}
\usepackage{parskip}

\babelprovide[import=el, main, onchar=ids fonts]{greek} % can also do import=el-polyton
\babelprovide[import, onchar=ids fonts]{english}

\babelfont{rm}
          [Language=Default]{Liberation Sans}
\babelfont[english]{rm}
          [Language=Default]{Liberation Sans}
\babelfont{sf}
          [Language=Default]{Liberation Sans}
\babelfont{tt}
          [Language=Default]{Liberation Sans}

%Enter Title Here
 \title{Team-risk-assessment-v0.2\\ LibShare}
\author{\textbf{Ονόματα / ΑΜ / Έτος:} \\ Γρηγόρης Καπαδούκας / 1072484 / 4\textdegree \\ Χρήστος Μπεστητζάνος / 1072615 / 4\textdegree \\ Νικόλαος Αυγέρης / 1067508 / 5\textdegree \\ Περικλής Κοροντζής / 1072563 / 4\textdegree}

\begin{document}

\makeatletter
\begin{center}
	\LARGE{\@title} \\
	\pagebreak
    \begin{LARGE}\@author\end{LARGE}
    \pagebreak
\end{center}

%Insert Body Here
\section{Ανάλυση Κινδύνων Ομάδας}

\textbf{Σημείωση:} Για την ανάλυση ρίσκων χρησιμοποιήσαμε τα πεδία που θεωρήσαμε ότι εφαρμόζουν από τη φόρμα καταγραφής κινδύνων στις διαφάνειες του μαθήματος. Όταν η τεχνική αντιμετώπισης και μετριασμού συμπίπτουν τις έχουμε βάλει μαζί.

\subsection*{Κίνδυνος 1:}
\begin{itemize}
	\item \textbf{Περιγραφή:} Έλλειψη συχνής επικοινωνίας. Τα μέλη της ομάδας διαμένουν σε διαφορετικές περιοχές και συχνά διαθέτουν περιορισμένο χρόνο.
	\item \textbf{Πιθανότητα Κινδύνου:} Μικρή
	\item \textbf{Συνδεόμενη Δραστηριότητα:} Όλες
	\item \textbf{Υπεύθυνος Αντιμετώπισης:} Όλη η ομάδα
	\item \textbf{Επίπεδο Σοβαρότητας Συνεπειών:} Υψηλό
	\item \textbf{Πρώτο Γεγονός Ενεργοποίησης του Κινδύνου:} Δυσκολία επικοινωνίας, ακύρωση ομαδικών συναντήσεων
	\item \textbf{Προτεραιότητα:} 1
	\item \textbf{Τρέχουσα Κατάσταση:} Ενεργή
	\item \textbf{Στρατηγική Αντιμετώπισης / Μετριασμού Κινδύνου:} Τα μέλη της ομάδας είναι  πολύ εξοικειωμένα με ηλεκτρονικούς τρόπους επικοινωνίας και έτσι μπορούν να φτιάξουν κανάλια επικοινωνίας και να είναι διαρκώς διαθέσιμα μεταξύ τους. 
\end{itemize}

\subsection*{Κίνδυνος 2:}
\begin{itemize}
	\item \textbf{Περιγραφή:} Έλλειψη ποιοτικής επικοινωνίας. Η επικοινωνία από ηλεκτρονικά μέσα, δυσκολεύουν την επίτευξη ποιοτικής επικοινωνίας και συχνά αυτό φέρνει σημαντικές καθυστερήσεις στα έργα λογισμικού.  
	\item \textbf{Πιθανότητα Κινδύνου:} Μεγάλη
	\item \textbf{Συνδεόμενη Δραστηριότητα:} Όλες
	\item \textbf{Υπεύθυνος Αντιμετώπισης:} Όλη η ομάδα
	\item \textbf{Επίπεδο Σοβαρότητας Συνεπειών:} Υψηλό
	\item \textbf{Πρώτο Γεγονός Ενεργοποίησης του Κινδύνου:} Δυσκολία συνεννόησης και συχνές παρανοήσεις όσον αφορά τις αναθέσεις εργασίας.
	\item \textbf{Προτεραιότητα:} 1
	\item \textbf{Τρέχουσα Κατάσταση:} Ενεργή
	\item \textbf{Στρατηγική Αντιμετώπισης / Μετριασμού Κινδύνου:} Λεπτομερής εξέταση του προγράμματος των μελών της ομάδας και εύρεση κοινού ελεύθερου χρόνου για συναντήσεις της ομάδας εβδομαδιαία με φυσική παρουσία. Επίσης για όσα μέλη μπορούν να συναντιούνται παραπάνω φορές μέσα στην εβδομάδα αυτό συνίσταται ανεπίσημα.

Αν δεν γίνεται να βρεθεί χρυσή τομή και κοινές ώρες, τα μέλη θα συναντιόνται μεταξύ τους σε μικρές ομάδες.
\end{itemize}

\subsection*{Κίνδυνος 3:}
\begin{itemize}
	\item \textbf{Περιγραφή:} Επειδή στην ομάδα όλες οι σημαντικές αποφάσεις έχουμε αποφασίσει να λαμβάνονται συλλογικά, πολλές φορές καθίσταται αναγκαία η χρήση ψηφοφορίας. Γεγονός που πολλές φορές μπορεί να οδηγήσει σε αδιέξοδα και μη δημιουργικές λύσεις λόγω φαινομένου αγελαίας σκέψης.
	\item \textbf{Πιθανότητα Κινδύνου:} Μεγάλη
	\item \textbf{Συνδεόμενη Δραστηριότητα:} Όλες
	\item \textbf{Υπεύθυνος Αντιμετώπισης:} Όλη η ομάδα
	\item \textbf{Επίπεδο Σοβαρότητας Συνεπειών:} Μεσαίο
	\item \textbf{Πρώτο Γεγονός Ενεργοποίησης του Κινδύνου:} Δυσκολία απόφασης για τη δομή ενός παραδοτέου.
	\item \textbf{Προτεραιότητα:} 3
	\item \textbf{Τρέχουσα Κατάσταση:} Ενεργή
	\item \textbf{Στρατηγική Αντιμετώπισης / Μετριασμού Κινδύνου:} Δεκτικότητα, σεβασμός και αλληλοκατανόηση μεταξύ των μελών. Υπάρχει επίσης δυνατότητα σε διάφορες αποφάσεις να βοηθήσει η γνώμη ενός "εξωτερικού ειδικού" (όπως του καθηγητή για παράδειγμα).
\end{itemize}

\subsection*{Κίνδυνος 4:}
\begin{itemize}
	\item \textbf{Περιγραφή:}Κάποια μέλη της ομάδας έχουν αρκετά μαθήματα που ασχολούνται ή εργάζονται πολλές ώρες. Έτσι είναι πιθανό να δυσκολεύονται να τηρούν τις προθεσμίες. 
	\item \textbf{Πιθανότητα Κινδύνου:} Μεσαία
	\item \textbf{Συνδεόμενη Δραστηριότητα:} Όλες
	\item \textbf{Υπεύθυνος Αντιμετώπισης:} Όλη η ομάδα
	\item \textbf{Επίπεδο Σοβαρότητας Συνεπειών:} Υψηλό
	\item \textbf{Πρώτο Γεγονός Ενεργοποίησης του Κινδύνου:} Μη έτοιμα παραδοτέα λίγο πριν την προθεσμία
	\item \textbf{Προτεραιότητα:} 1
	\item \textbf{Τρέχουσα Κατάσταση:} Ανενεργή
	\item \textbf{Στρατηγική Αντιμετώπισης / Μετριασμού Κινδύνου:} Συζήτηση και προγραμματισμός βασισμένος πάνω στις ανάγκες της ομάδας. Συνεργασία σε δυάδες, ώστε σε ξαφνική αποχώρηση ενός ατόμου για κάποια χρονική περίοδο να μη σταματήσει το έργο.
\end{itemize}

\subsection*{Κίνδυνος 5:}
\begin{itemize}
	\item \textbf{Περιγραφή:} Το πρότζεκτ στα τελευταία παραδοτέα θα χρειαστεί συγγραφή κώδικα. Τότε όμως θα είναι περίοδος εξεταστικής για την ομάδα και δεν θα υπάρχει πολύς χρόνος για σωστή επανάληψη και ενασχόληση με γλώσσες προγραμματισμού διασύνδεση διαφορετικών πλατφορμών κτλ.
	\item \textbf{Πιθανότητα Κινδύνου:} Μεσαία
	\item \textbf{Συνδεόμενη Δραστηριότητα:} ΤΥ 11, ΤΥ 12, ΤΥ 23 (από Team-plan v0.2 Σχήμα 4)
	\item \textbf{Υπεύθυνος Αντιμετώπισης:} Όλη η ομάδα
	\item \textbf{Επίπεδο Σοβαρότητας Συνεπειών:} Μεσαίο
	\item \textbf{Πρώτο Γεγονός Ενεργοποίησης του Κινδύνου:} Έλλειψη εμπειρίας τεχνικών δεξιοτήτων.
	\item \textbf{Προτεραιότητα:} 2
	\item \textbf{Τρέχουσα Κατάσταση:} Ενεργή
	\item \textbf{Στρατηγική Αντιμετώπισης / Μετριασμού Κινδύνου:} Απόφαση από νωρίς για τις γλώσσες προγραμματισμού και τις τεχνικές ικανότητες που θα χρειαστούνε και σταδιακή εξοικείωση κάθε ατόμου με κάποιο χρήσιμο κομμάτι.
\end{itemize}

\section{Συμμετοχή και Ρόλοι στη Συγγραφή του Κειμένου}
\begin{enumerate}
	\item \textbf{Νικόλαος Αυγέρης:} Author
	\item \textbf{Γρηγόρης Καπαδούκας:} Peer Reviewer
	\item \textbf{Χρήστος Μπεστητζάνος:} Peer Reviewer
\end{enumerate}

\section{Αλλαγές από έκδοση σε έκδοση}

\subsection{Από έκδοση v0.1 σε έκδοση v0.2}
\begin{itemize}
    \item Διόρθωση των συνδεόμενων δραστηριοτήτων ώστε να συμπίπτουν με τα ΤΥ του Team-plan v0.2 αντί για το v0.1.
\end{itemize}


\end{document}
