%! TeX program = lualatex
\documentclass[12pt,a4paper]{article}

\usepackage[nil]{babel}
\usepackage{unicode-math}
\usepackage[svgnames]{xcolor}
\usepackage{lmodern}
\usepackage{graphicx}
\usepackage{wrapfig}
\usepackage{float}
\usepackage{parskip}
\usepackage{hyperref}
\usepackage{listings}

\definecolor{codegreen}{rgb}{0,0.6,0}
\definecolor{codegray}{rgb}{0.5,0.5,0.5}
\definecolor{codepurple}{rgb}{0.58,0,0.82}
\definecolor{backcolour}{rgb}{0.95,0.95,0.92}

\lstdefinestyle{mystyle}{
    backgroundcolor=\color{backcolour},   
    commentstyle=\color{codegreen},
    keywordstyle=\color{magenta},
    stringstyle=\color{codepurple},
    basicstyle=\ttfamily\footnotesize,
    breakatwhitespace=false,         
    breaklines=true,                 
    captionpos=b,                    
    keepspaces=true,                 
    showspaces=false,                
    showstringspaces=false,
    showtabs=false,                  
    tabsize=2
}


\lstset{style=mystyle}


\babelprovide[import=el, main, onchar=ids fonts]{greek} % can also do import=el-polyton
\babelprovide[import, onchar=ids fonts]{english}

\babelfont{rm}
          [Language=Default]{Liberation Sans}
\babelfont[english]{rm}
          [Language=Default]{Liberation Sans}
\babelfont{sf}
          [Language=Default]{Liberation Sans}
\babelfont{tt}
          [Language=Default]{Liberation Sans}

%Enter Title Here
 \title{Project-code-v1.0 \\ LibShare}
 \author{\textbf{Ονόματα / ΑΜ / Έτος:} \\ Γρηγόρης Καπαδούκας / 1072484 / 4\textdegree \\ Χρήστος Μπεστητζάνος / 1072615 / 4\textdegree \\ Νικόλαος Αυγέρης / 1067508 / 5\textdegree \\ Περικλής Κοροντζής / 1072563 / 4\textdegree\\ \href{https://github.com/GregKapadoukas/University-Software-Engineering-Project}{\color{blue}GitHub Link}}

\begin{document}

\makeatletter
\begin{center}
	\LARGE{\@title} \\
	\pagebreak
    \begin{LARGE}\@author\end{LARGE}
    \pagebreak
\end{center}

%Insert Body Here
\section{Σύνδεσμος GitHub με Κώδικα}
Ο κώδικας που γράφτηκε για την εργασία βρίσκεται στο repository στον παρακάτω σύνδεσμο που οδηγεί απευθείας στο κώδικα της εργασίας:

\textcolor{blue}{\href{https://github.com/GregKapadoukas/University-Software-Engineering-Project/tree/master/Code}{https://github.com/GregKapadoukas/University-Software-Engineering-\\Project/tree/master/Code}}

\section{Τι Περιλαμβάνει η Έκδοση v1.0}

Η έκδοση v1.0 του Project-code περιλαμβάνει όλο το κώδικα με όλη τη λειτουργικότητα από όλα τα use cases που σχεδιάσαμε. Η μόνη λειτουργικότητα που λείπει είναι η προβολή της εικόνας χρήστη και η εναλλακτική ροή της αλλαγής εικόνας χρήστη. Screenshots του κώδικα περιέχονται επίσης στο τεχνικό κείμενο Project-description-v1.0.pdf, όπως μας ζητείται.

\section{Σημειώσεις Σχετικές με τον Κώδικα}
\label{Notes}

\begin{itemize}
    \item Αρχικά αναφέρεται ότι δεν γίνεται χρήση βάσης δεδομένων για την αποθήκευση των δεδομένων του προγράμματος. Τα δεδομένα έχουν τη μορφή στιγμιοτύπων των κλάσεων που έχουμε δημιουργήσει και γίνεται προσπέλαση αυτών μέσω της ανάγνωσης των τιμών των πεδίων των στιγμιοτύπων (μέσω αντίστοιχων getter methods). Οπότε τα νέα δεδομένα (στιγμιότυπα) που προστίθενται ή ανανεώνονται στην εφαρμογή κατά την εκτέλεση δεν είναι persistent από εκτέλεση σε εκτέλεση.
    \item Πολλές φορές κατά τη χρήση του προγράμματος, όταν γίνεται προσθήκη, αλλαγή ή διαγραφή σε δεδομένα είναι αναγκαία η ανανέωση του frame της σελίδας ώστε να νέα δεδομένα να εμφανιστούν. Αυτό γίνεται μέσω κλικ της οθόνης που βρίσκεται ήδη ο χρήστης ξανά στο navigation bar.
    \item Τέλος, επειδή δεν έχει υλοποιηθεί σύστημα log in και log out, εφόσον δεν αναλύσαμε κάτι τέτοιο στα use cases, θεωρούμε ότι γίνεται αυτόματο log in ενός χρήστη που ονομάζουμε currentUser. Στη περίπτωση όμως της ανανέωσης της κατάστασης μιας συναλλαγής στο Dashboard ('Accept' ή 'Deny' transaction, 'Mark Delivered', 'Mark Returned') που χρειάζεται αλληλεπίδραση μεταξύ δύο χρηστών, θεωρούμε πως ο χρήστης με τον οποίο αλληλεπιδρά ο currentUser κάνει αυτόματα 'Accept' τη συναλλαγή όταν τη δημιουργεί ο currentUser, αυτόματα 'Mark Delivered' αφού το κάνει και ο currentUser και αυτόματα 'Mark Returned' αφού το κάνει και ο curreentUser. Με αυτό το τρόπο αποφεύγουμε το log out και log in ως τον άλλο χρήστη, παρόλα αυτά η λειτουργικότητα του κώδικα έχει συγγραφτεί ώστε να υπάρχει η λειτουργία από πίσω για όλες τις πιθανές ενέργειες που θα επιθυμούσε να κάνει ο άλλος χρήστης.
    \item Σχετικά με την σελίδα αναζήτησης, σημειώνουμε πως για αναζήτηση βιβλίου και αίτησης γίνεται substring matching στους τίτλους και τους συγγραφείς των διαθέσιμων βιβλίων. Άρα για κενή αναζήτηση, το string '' (κενό) είναι substring όλων των string, οπότε θα εμφανιστούν όλα τα διαθέσιμα βιβλία και αιτήσεις. Αντιθέτως για αναζήτηση χρήστη, πρέπει να εισαχθεί το πλήρη username του άλλου χρήστη για να εμφανιστεί η σελίδα του (string matching) άρα προτείνεται να δοκιμαστούν τα search terms 'Xristos' και 'George' για παράδειγμα, που αναπαριστούν χρήστες που έχουμε προσθέσει ως στιγμιότυπα 'User' στον κώδικα.
\end{itemize}

\section{Οδηγίες Εκτέλεσης του Κώδικα και Εγκατάσταση Dependencies}
Για την διεκπεραίωση αυτής της εργασίας έχουμε επιλέξει να χρησιμοποιήσουμε γλώσσα προγραμματισμού Python μαζί τη βιβλιοθήκη CustomTKinter για το GUI και Matplotlib για το UI.

\begin{enumerate}
    \item Αρχικά λοιπόν πρέπει να εγκατασταθεί η Python από την official σελίδα τους ή από κάποιο package manager. Η έκδοση που χρησιμοποιήσαμε για την συγγραφή του κώδικα είναι η 3.11.3.

    \item Έπειτα πρέπει να εγκατασταθούν οι βιβλιοθήκες CustomTKinter,\\Matplotlib και CTKMessageBox μέσω της εξής εντολής σε τερματικό ή cmd:

\begin{lstlisting}[language=Bash]
pip install customtkinter matplotlib ctkmessagebox\end{lstlisting}

    \item Έπειτα εκτελούμε τον κώδικα μέσω της main, όπως φαίνεται παρακάτω:

\begin{lstlisting}[language=Bash]
python main.py\end{lstlisting}
\end{enumerate}

\textbf{Εναλλακτικά:}

Γίνεται να χρησιμοποιηθεί το distribution Minicoda για τη δημιουργία ενός Python virtual environment στο οποίο θα εκτελεστεί ο κώδικας. Αυτή είναι η προσέγγιση που προτιμήσαμε και εμείς και προτείνουμε. Τα βήματα είναι τα εξής:

\begin{enumerate}
    \item Εγκατάσταση του Miniconda μέσω του installer στη σελίδα:

        \textcolor{blue}{\href{https://docs.conda.io/en/latest/miniconda.html}{https://docs.conda.io/en/latest/miniconda.html}}
    \item Δημιουργία ενός conda virtual environment που χρησιμοποιεί την έκδοση 3.11.3 της Python που χρησιμοποιούμε και εμείς μέσω της εξής εντολής:

\begin{lstlisting}[language=Bash]
conda create -n ctk python=3.11.3\end{lstlisting}

    \item Ενεργοποίηση του conda environment μέσω της εξής εντολής:

\begin{lstlisting}[language=Bash]
conda activate ctk\end{lstlisting}

    \item Εγκατάσταση του CustomTKinter μέσω της εξής εντολής:

\begin{lstlisting}[language=Bash]
pip install customtkinter matplotlib, ctkmessagebox\end{lstlisting}

    \item Εκτέλεση του κώδικα μέσω της παρακάτω εντολής:
\begin{lstlisting}[language=Bash]
python main.py\end{lstlisting}

\end{enumerate}

\section{Συμμετοχή και Ρόλοι στη Συγγραφή του Κειμένου}
\begin{enumerate}
	\item \textbf{Γρηγόρης Καπαδούκας:} Συγγραφέας τεχνικού κειμένου.
    \item Η συνεισφορά μας στη συγγραφή του κώδικα φαίνεται στο GitHub της εργασίας μέσω του ιστορικού των commits.
\end{enumerate}

\section{Αλλαγές από έκδοση σε έκδοση}

\subsection{Από έκδοση v0.1 σε έκδοση v0.2}
\begin{itemize}
    \item Σχολιασμός του τι έχει γίνει στο κώδικα για την έκδοση v0.2, ανανέωση του συνδέσμου.
    \item Ανανέωση των οδηγιών ώστε να εγκατασταθούν και οι νέες βιβλιοθήκες που αποφασίσαμε να χρησιμοποιήσουμε.
\end{itemize}

\subsection{Από έκδοση v0.2 σε έκδοση v1.0}
\begin{itemize}
    \item Σχολιασμός του τι έχει γίνει στο κώδικα για την έκδοση v1.0, ανανέωση του συνδέσμου ώστε να δείχνει στο κώδικα του τελευταίου commit του master branch στο GitHub της εργασίας.
    \item Προσθήκη του κεφαλαίου \ref{Notes}: "Σημειώσεις Σχετικές με τον Κώδικα".
\end{itemize}


\end{document}
