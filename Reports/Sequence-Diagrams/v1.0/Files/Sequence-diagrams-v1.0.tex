%! TeX program = lualatex
\documentclass[12pt,a4paper]{article}

\usepackage[nil]{babel}
\usepackage{unicode-math}
\usepackage[svgnames]{xcolor}
\usepackage{lmodern}
\usepackage{graphicx}
\usepackage{wrapfig}
\usepackage{float}
\usepackage{parskip}

\babelprovide[import=el, main, onchar=ids fonts]{greek} % can also do import=el-polyton
\babelprovide[import, onchar=ids fonts]{english}

\babelfont{rm}
          [Language=Default]{Liberation Sans}
\babelfont[english]{rm}
          [Language=Default]{Liberation Sans}
\babelfont{sf}
          [Language=Default]{Liberation Sans}
\babelfont{tt}
          [Language=Default]{Liberation Sans}

%Enter Title Here
 \title{Sequence-diagrams-v1.0 \\ LibShare}
\author{\textbf{Ονόματα / ΑΜ / Έτος:} \\ Γρηγόρης Καπαδούκας / 1072484 / 4\textdegree \\ Χρήστος Μπεστητζάνος / 1072615 / 4\textdegree \\ Νικόλαος Αυγέρης / 1067508 / 5\textdegree \\ Περικλής Κοροντζής / 1072563 / 4\textdegree}

\begin{document}

\makeatletter
\begin{center}
	\LARGE{\@title} \\
	\pagebreak
    \begin{LARGE}\@author\end{LARGE}
    \pagebreak
\end{center}

%Insert Body Here
\section{Sequence Diagrams}

\textbf{Σημείωση έκδοσης v1.0:}

Θέλαμε εδώ να επισημάνουμε ξανά, όπως κάναμε και στο τεχνικό κείμενο των robustness-diagrams-v1.0, ότι σε πολλές περιπτώσεις στα robustness diagrams λόγω της δομής τους αναγκαστήκαμε πολλές οθόνες που θέλαμε να είναι μια να τις δείξουμε ως πολλά συνοριακά αντικείμενα. Το ίδιο ισχύει και για τα αντικείμενα οντοτήτων, που όταν χρησιμοποιούνται πολλά στιγμιότυπα μιας τελικής κλάσεις, για λόγους συνέχειας αναπαραστάθηκαν τα στιγμιότυπα της ίδιας τελικής κλάσης ως ξεχωριστά αντικείμενα οντοτήτων.

Οπότε για τους παραπάνω λόγους καταλήξαμε να μεταφέρουμε μόνο ένα από τα συνοριακά αντικείμενα και τα αντικείμενα οντοτήτων των robustness diagrams όταν αυτά ήταν διπλότυπα στα sequence diagrams, και να φτιάχνουμε μόνο μια κλάση στο class diagram και στον κώδικα.

\subsection{Αναζήτηση βιβλίων / χρήστη / αιτήσεων}
\begin{figure}[H]
	\includegraphics[width=\textwidth]{Search Sequence.png}
	\caption{Sequence Diagram: Αναζήτηση βιβλίων / χρήστη / αιτήσεων}
	\label{Sequence Diagram: Αναζήτηση βιβλίων / χρήστη / αιτήσεων}
\end{figure}

\subsection{Αποδοχή Προσφοράς Βιβλίου και Ενοικίαση}
\label{Rent}

\begin{figure}[H]
	\includegraphics[width=\textwidth]{Accept Book Offer and Rent Sequence.png}
	\caption{Sequence Diagram: Αποδοχή Προσφοράς Βιβλίου και Ενοικίαση}
	\label{Sequence Diagram: Αποδοχή Προσφοράς Βιβλίου και Ενοικίαση}
\end{figure}


Εδώ αναφέρεται ότι η σειρά με την οποία οι δύο χειριστές επιλέγουν να μεταβούν μέσω του GUI στο SearchPage (μέσω του show\_Frame()) καθώς και η σειρά με την οποία γίνεται η ενημέρωση παραλαβής και επιστροφής (μέσω των updateStatus()) δεν έχει σημασία για το σύστημα. Το μόνο που ελέγχεται από το σύστημα είναι ότι και οι δύο χρήστες πρέπει να κάνουν ενημέρωση παραλαβής και μετά να κάνουν και οι δύο πάλι ενημέρωση επιστροφής.

\subsection{Αποδοχή Αίτησης Βιβλίου και Ενοικίαση}
\begin{figure}[H]
	\includegraphics[width=\textwidth]{Accept Request Offer and Rent Sequence.png}
	\caption{Sequence Diagram: Αποδοχή Αίτησης Βιβλίου και Ενοικίαση}
	\label{Sequence Diagram: Αποδοχή Αίτησης Βιβλίου και Ενοικίαση}
\end{figure}

Στο sequence diagram για τη "Αποδοχή Αίτησης Βιβλίου και Ενοικίαση" ισχύουν οι ίδιες σημειώσεις και παραδοχές που ίσχυαν και για το sequence diagram στο κεφάλαιο \ref{Rent}, εφόσον τα δύο use cases "Αποδοχή Αίτησης Βιβλίου και Ενοικίαση" και "Αποδοχή Προσφοράς Βιβλίου και Ενοικίαση" αποτελούν δύο πτυχές της ίδιας διαδικασίας, που εξηγήσαμε στο τεχνικό κείμενο "Use-cases-v1.0" γιατί τα αναφέρουμε ξεχωριστά.

\subsection{Διαχείριση των βιβλίων που προσφέρει ο χρήστης προς ενοικίαση από άλλους}
\begin{figure}[H]
	\includegraphics[width=\textwidth]{Manage User Book Listings Sequence.png}
	\caption{Sequence Diagram: Διαχείριση των βιβλίων που προσφέρει ο χρήστης προς ενοικίαση από άλλους}
	\label{Sequence Diagram: Διαχείριση των βιβλίων που προσφέρει ο χρήστης προς ενοικίαση από άλλους}
\end{figure}

Σημειώνουμε εδώ ότι στο κώδικα έχουμε κάνει χρήση πολλών getters για να εμφανίσουμε πληροφορίες στο UI, που δεν τις εμφανίζουμε εδώ για λόγους απλότητας.

\subsection{Διαχείριση των αιτήσεων που έχει κάνει ο χρήστης}

\begin{figure}[H]
	\includegraphics[width=\textwidth]{Manage User Requests Sequence.png}
	\caption{Sequence Diagram: Διαχείριση των αιτήσεων που έχει κάνει ο χρήστης}
	\label{Sequence Diagram: Διαχείριση των αιτήσεων που έχει κάνει ο χρήστης}
\end{figure}

Σημειώνουμε εδώ ότι στο κώδικα έχουμε κάνει χρήση πολλών getters για να εμφανίσουμε πληροφορίες στο UI, που δεν τις εμφανίζουμε εδώ για λόγους απλότητας.

\subsection{Αξιολόγηση άλλων χρηστών μετά από την ολοκλήρωση συναλλαγής}
\begin{figure}[H]
	\includegraphics[width=\textwidth]{Review after Transaction Sequence.png}
	\caption{Sequence Diagram: Αξιολόγηση άλλων χρηστών μετά από την ολοκλήρωση συναλλαγής}
	\label{Sequence Diagram: Αξιολόγηση άλλων χρηστών μετά από την ολοκλήρωση συναλλαγής}
\end{figure}


\subsection{Προβολή και Επεξεργασία στοιχείων λογαριασμού χρήστη}
\begin{figure}[H]
	\includegraphics[width=\textwidth]{View and Edit User Account Details Sequence.png}
	\caption{Sequence Diagram: Προβολή και Επεξεργασία στοιχείων λογαριασμού χρήστη}
	\label{Sequence Diagram: Προβολή και Επεξεργασία στοιχείων λογαριασμού χρήστη}
\end{figure}

\subsection{Επεξεργασία χρηματικού υπολοίπου χρήστη}
\begin{figure}[H]
	\includegraphics[width=\textwidth]{Edit User Balance Sequence.png}
	\caption{Sequence Diagram: Επεξεργασία χρηματικού υπολοίπου χρήστη}
	\label{Sequence Diagram: Επεξεργασία χρηματικού υπολοίπου χρήστη}
\end{figure}

\subsection{Επεξεργασία λίστας αγαπημένων και χρήση συστήματος ειδοποιήσεων}
\begin{figure}[H]
	\includegraphics[width=\textwidth]{Favorite Users and Notification System Sequence.png}
	\caption{Sequence Diagram: Επεξεργασία λίστας αγαπημένων και χρήση συστήματος ειδοποιήσεων}
	\label{Sequence Diagram: Επεξεργασία λίστας αγαπημένων και χρήση συστήματος ειδοποιήσεων}
\end{figure}

\subsection{Προβολή ιστορικού συναλλαγών και στατιστικών}
\begin{figure}[H]
	\includegraphics[width=\textwidth]{History and Statistics Sequence.png}
	\caption{Sequence Diagram: Προβολή ιστορικού συναλλαγών και στατιστικών}
	\label{Sequence Diagram: Προβολή ιστορικού συναλλαγών και στατιστικών}
\end{figure}

\section{Συμμετοχή και Ρόλοι στη Συγγραφή του Κειμένου}
\begin{enumerate}
	\item \textbf{Γρηγόρης Καπαδούκας:} Author (Κεφαλαίων 1.1, 1.2, 1.3), Editor (Κεφαλαίων 1.9), Reviewer Όλων
	\item \textbf{Χρήστος Μπεστητζάνος:} Author (Κεφαλαίων 1.5, 1.6), Editor (Κεφαλαίων 1.4, 1.7, 1.8, 1.10), Reviewer Όλων
   	\item \textbf{Νικόλαος Αυγέρης:} Author (Κεφαλαίων 1.7, 1.8, 1.9)
	\item \textbf{Περικλής Κοροντζής:} Author (Κεφαλαίων 1.4, 1.10)
\end{enumerate}

\section{Αλλαγές από έκδοση σε έκδοση}

\subsection{Από έκδοση v0.1 σε έκδοση v1.0}
\begin{itemize}
    \item Διόρθωση όλων των Sequence Diagrams, ώστε αυτές πλέον να αναπαριστούν κλάσεις και μάλιστα αυτές του κώδικα.
    \item Αφαίρεση lifeline objects με μορφή boundary object και entity objects από τα πρώτα τρια sequence.
    \item Ανανέωση των sequence ώστε να υπάρχει αντιστοιχία με τον κώδικα.
\end{itemize}

\end{document}
